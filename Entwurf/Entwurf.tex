\documentclass[a4paper,12pt]{article}
\usepackage{amssymb} % needed for math
\usepackage{amsmath} % needed for math
\usepackage[utf8]{inputenc} % this is needed for german umlauts
\usepackage[ngerman]{babel} % this is needed for german umlauts
\usepackage[T1]{fontenc}    % this is needed for correct output of umlauts in pdf
\usepackage[margin=2.5cm]{geometry} %layout
\usepackage{booktabs}
\usepackage[hidelinks]{hyperref}
\hypersetup{pdftitle={Balanced Banana},bookmarks=true,}
\usepackage{graphicx}
\usepackage{csquotes}
\usepackage[nonumberlist]{glossaries}
\usepackage{enumitem}
\usepackage{verbatim}
\usepackage{indentfirst} % Adds indent for the first paragraph after a {/section}


\deftranslation[to=ngerman]{Glossary}{\section{Stichwortverzeichnis}}

\makeatletter
\newenvironment{mycode}
 {\def\@xobeysp{\ }\verbatim\rightskip=0pt plus 6em\relax}
 {\endverbatim}
\makeatother

\setitemize{align=parleft, labelsep=0.5cm}


\makenoidxglossaries



\title{Balanced Banana}
\author{Niklas Lorenz \and Thomas Häuselmann \and Rakan Zeid Al Masri \and Christopher Lukas Homberger \and Jonas Seiler}


%%%%%%%%%%%%%%%%%%%%%%%%%%%%%%%%%%%%%%%%%%%%%%%%%%%%%%%%%%%%%%%%%%%%%%
% Create a shorter version for tables. DO NOT CHANGE               	 %
%%%%%%%%%%%%%%%%%%%%%%%%%%%%%%%%%%%%%%%%%%%%%%%%%%%%%%%%%%%%%%%%%%%%%%
\newcommand\addrow[2]{#1 &#2\\ }

\newcommand\addheading[2]{#1 &#2\\ \hline}
\newcommand\tabularhead{\begin{tabular}{lp{13cm}}
\hline
	}

\newcommand\addmulrow[2]{ \begin{minipage}[t][][t]{2.5cm}#1\end{minipage}%
   &\begin{minipage}[t][][t]{8cm}
    \begin{enumerate} #2   \end{enumerate}
    \end{minipage}\\ }

\newenvironment{usecase}{\tabularhead}
{\hline\end{tabular}}

\usepackage{microtype}

\begin{document}
\pagenumbering{roman}
\begin{titlepage}
    \begin{center}
    
     \vspace*{0.8cm}
 
        \includegraphics[width=0.5\textwidth]{balancedbanana}
        \vspace*{1cm}
 
        \Huge
        \textbf{Balanced Banana}
 
        \vspace{0.5cm}
        \LARGE
        A Distributed Task Scheduling System
        
        \vspace{0.5 cm}
        \LARGE
        Implementierung
 
        \vspace{1.5cm}

        \large
        \textbf{Niklas Lorenz, Thomas Häuselmann, Rakan Zeid Al Masri, Christopher Lukas Homberger und Jonas Seiler}
 
        \vspace*{0.5cm}

        \textbf{\today}
 
       
        
 
    \end{center}
\end{titlepage}         % Deckblatt.tex laden und einfügen
\setcounter{page}{2}
\tableofcontents          % Inhaltsverzeichnis ausgeben
\clearpage
\pagenumbering{arabic}

% Document starts here.
\section{Einleitung}
\vspace{1cm}

% Here

\clearpage
\section{Aufbau}


\clearpage
\section{Klassenbeschreibung}

\iffalse
Format:
\subsubsection{Klasse}

Kurze Beschreibung

\begin{itemize}[label={}]

	\item \textit{\textbf{Attribute}}
		\begin{itemize}[label={\textbullet}]
			\item \textit{name} beschreibung
		\end{itemize}

	\item \textit{\textbf{Methoden}}
		\begin{itemize}[label={\textbullet}]
			\item \textit{signatur} beschreibung
		\end{itemize}


\end{itemize}

\fi
\subsection{Datenbank}
\subsubsection{Model}
%Show the class diagram for it first

\subsubsection{Repository}

Die Repository-Klasse ist die Schnittstelle, die der Rest des Programms verwendet, um SQL-Queries durchzuführen und deren Ergebnisse zu interpretieren. Es verwendet die Gateway-Klasse, um die SQL-Queries durchzuführen und erzeugt ein Objekt, indem es der Factory-Klasse die von Gateway zurückgegebenen Daten gibt.

	\begin{itemize}[label={}]
	
		\item \textit{\textbf{Methoden}}
			\begin{itemize}[label={\textbullet}]
				\item \textit{public int addWorker(int auth\_key, int space, int ram, int cores, std::string address)} Fügt einen Worker zur Datenbank hinzu und gibt seine ID zurück.
				
				\item \textit{public bool removeWorker(int id)} Löscht einen Worker aus dem DB. Gibt true zurück, wenn die Operation erfolgreich war, ansonsten false.
				
				\item \textit{public Worker getWorker(int worker\_id)} Gibt den Worker mit der angegebenen ID zurück.
				
				\item \textit{public std::vector<std::shared\_ptr<Worker>> getWorkers()} Gibt alle Workers zurück.
				
				\item \textit{public int addJob(int user\_id, JobConfig config, std::string schedule\_time, std::string command)} Fügt einen Job zur Datenbank hinzu und gibt seine ID zurück. 
				
				\item \textit{public bool removeJob(int job\_id)} Löscht einen Job aus dem DB. Gibt true zurück, wenn die Operation erfolgreich war, ansonsten false.
				
				\item \textit{public Job getJob(int job\_id)} Gibt den Job mit der angegebenen ID zurück.
				
				\item \textit{public std::vector<std::shared\_ptr<Job>> getJobs()} Gibt alle Jobs zurück.
				
				\item \textit{public int addUser(std::string name, std::string email, int auth\_key)} Fügt einen User zur Datenbank hinzu und gibt seine ID zurück. 
				
				\item \textit{public bool removeUser(int user\_id)} Löscht einen User aus dem DB. Gibt true zurück, wenn die Operation erfolgreich war, ansonsten false.
				
				\item \textit{public User getUser(int user\_id)} Gibt den User mit der angegebenen ID zurück.
				
				\item \textit{public std::vector<std::shared\_ptr<User>> getUsers()} Gibt alle Users zurück.
				
				\item \textit{public bool startJob(int job\_id, int worker\_id, specs specs, std::string start\_time)} Aktualisiert den Eintrag eines Jobs in der Datenbank mit einer Startzeit, den zugewiesenen Ressourcen und dem zugeordneten Mitarbeiter. Gibt true zurück, wenn die Operation erfolgreich war, ansonsten false.
				
				\item \textit{public bool finishJob(int job\_id, std::string finish\_time, std::string stdout, int exit\_code)} Aktualisiert den Eintrag eines Jobs mit Endzeit, Ausgabe und Exitcode. Gibt true zurück, wenn die Operation erfolgreich war, ansonsten false.
				
				\item \textit{public job\_result getJobResult(int job\_id)} Gibt die Ergebnisse eines \textbf{fertigen} Jobs zurück.
								
			\end{itemize}
			
	\end{itemize}
\clearpage
\subsubsection{Gateway}

Die Gateway-Klasse verbindet sich mit der Datenbank und führt die SQL-Queries aus.

\begin{itemize}[label={}]

	\item \textit{\textbf{Attribute}}
		\begin{itemize}[label={\textbullet}]
			\item \textit{QSqlDatabase db} Verwaltet die Verbindung zur Datenbank. 
		\end{itemize}

	\item \textit{\textbf{Methoden}}
		\begin{itemize}[label={\textbullet}]
			\item \textit{public int addWorker(int auth\_key, int space, int ram, int cores, std::string address)} Fügt einen Worker zur Datenbank hinzu und gibt seine ID zurück.
				
				\item \textit{public bool removeWorker(int id)} Löscht einen Worker aus dem DB. Gibt true zurück, wenn die Operation erfolgreich war, ansonsten false.
				
				\item \textit{public worker\_details getWorker(int worker\_id)} Gibt die Daten des Workers mit der angegebenen ID zurück.
				
				\item \textit{public std::vector<std::shared\_ptr<worker\_details>> getWorkers()} Gibt die Daten jedes Mitarbeiters zurück.
				
				\item \textit{public int addJob(int user\_id, JobConfig config, std::string schedule\_time, std::string command)} Fügt einen Job zur Datenbank hinzu und gibt seine ID zurück. 
				
				\item \textit{public bool removeJob(int job\_id)} Löscht einen Job aus dem DB. Gibt true zurück, wenn die Operation erfolgreich war, ansonsten false.
				
				\item \textit{public job\_details getJob(int job\_id)} Gibt die Daten des Jobs mit der angegebenen ID zurück.
				
				\item \textit{public std::vector<std::shared\_ptr<job\_details>> getJobs()} Gibt die Daten jedes Jobs zurück.
				
				\item \textit{public int addUser(std::string name, std::string email, int auth\_key)} Fügt einen User zur Datenbank hinzu und gibt seine ID zurück. 
				
				\item \textit{public bool removeUser(int user\_id)} Löscht einen User aus dem DB. Gibt true zurück, wenn die Operation erfolgreich war, ansonsten false.
				
				\item \textit{public user\_details getUser(int user\_id)} Gibt die Daten des Users mit der angegebenen ID zurück
				
				\item \textit{public std::vector<std::shared\_ptr<user\_details>> getUsers()} Gibt die Daten jedes Users zurück.
				
				\item \textit{public bool startJob(int job\_id, int worker\_id, specs specs, std::string start\_time)} Aktualisiert den Eintrag eines Jobs in der Datenbank mit einer Startzeit, den zugewiesenen Ressourcen und dem zugeordneten Mitarbeiter. Gibt true zurück, wenn die Operation erfolgreich war, ansonsten false.
				
				\item \textit{public bool finishJob(int job\_id, std::string finish\_time, std::string stdout, int exit\_code)} Aktualisiert den Eintrag eines Jobs mit Endzeit, Ausgabe und Exitcode. Gibt true zurück, wenn die Operation erfolgreich war, ansonsten false.
				
				\item \textit{public job\_result getJobResult(int job\_id)} Gibt die Ergebnisse eines \textbf{fertigen} Jobs zurück.
		\end{itemize}


\end{itemize}

\subsubsection{Factory}

Erzeugt Objekte aus gegebenen Daten.

\begin{itemize}[label={}]

	\item \textit{\textbf{Methoden}}
		\begin{itemize}[label={\textbullet}]
			\item \textit{public Job createJob(job\_details info)} Erzeugt ein Job-Objekt.
			
			\item \textit{public Worker createWorker(worker\_details info)} Erzeugt ein Worker-Objekt.
			
			\item \textit{public User createUser(user\_details info)} Erzeugt ein User-Objekt.
		\end{itemize}


\end{itemize}

% Document ends here.
\printnoidxglossaries

\end{document}
