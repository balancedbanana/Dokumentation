\documentclass[a4paper,12pt]{article}
\usepackage{amssymb} % needed for math
\usepackage{amsmath} % needed for math
\usepackage[utf8]{inputenc} % this is needed for german umlauts
\usepackage[ngerman]{babel} % this is needed for german umlauts
\usepackage[T1]{fontenc}    % this is needed for correct output of umlauts in pdf
\usepackage[margin=2.5cm]{geometry} %layout
\usepackage{booktabs}
\usepackage[hidelinks]{hyperref}
\hypersetup{pdftitle={Balanced Banana},bookmarks=true,}
\usepackage{graphicx}
\usepackage{csquotes}
\usepackage[nonumberlist]{glossaries}
\usepackage{enumitem}
\usepackage{verbatim}
\usepackage{indentfirst} % Adds indent for the first paragraph after a {/section}


\deftranslation[to=ngerman]{Glossary}{\section{Stichwortverzeichnis}}

\makeatletter
\newenvironment{mycode}
 {\def\@xobeysp{\ }\verbatim\rightskip=0pt plus 6em\relax}
 {\endverbatim}
\makeatother

\setitemize{align=parleft, labelsep=0.5cm}


\title{Balanced Banana}
\author{Niklas Lorenz \and Thomas Häuselmann \and Rakan Zeid Al Masri \and Christopher Lukas Homberger \and Jonas Seiler}


%%%%%%%%%%%%%%%%%%%%%%%%%%%%%%%%%%%%%%%%%%%%%%%%%%%%%%%%%%%%%%%%%%%%%%
% Create a shorter version for tables. DO NOT CHANGE               	 %
%%%%%%%%%%%%%%%%%%%%%%%%%%%%%%%%%%%%%%%%%%%%%%%%%%%%%%%%%%%%%%%%%%%%%%
\newcommand\addrow[2]{#1 &#2\\ }

\newcommand\addheading[2]{#1 &#2\\ \hline}
\newcommand\tabularhead{\begin{tabular}{lp{13cm}}
\hline
	}

\newcommand\addmulrow[2]{ \begin{minipage}[t][][t]{2.5cm}#1\end{minipage}%
   &\begin{minipage}[t][][t]{8cm}
    \begin{enumerate} #2   \end{enumerate}
    \end{minipage}\\ }

\newenvironment{usecase}{\tabularhead}
{\hline\end{tabular}}

\usepackage{microtype}

\begin{document}
\pagenumbering{roman}
\begin{titlepage}
    \begin{center}
    
     \vspace*{0.8cm}
 
        \includegraphics[width=0.5\textwidth]{balancedbanana}
        \vspace*{1cm}
 
        \Huge
        \textbf{Balanced Banana}
 
        \vspace{0.5cm}
        \LARGE
        A Distributed Task Scheduling System
        
        \vspace{0.5 cm}
        \LARGE
        Implementierung
 
        \vspace{1.5cm}

        \large
        \textbf{Niklas Lorenz, Thomas Häuselmann, Rakan Zeid Al Masri, Christopher Lukas Homberger und Jonas Seiler}
 
        \vspace*{0.5cm}

        \textbf{\today}
 
       
        
 
    \end{center}
\end{titlepage}         % Deckblatt.tex laden und einfügen
\setcounter{page}{2}
\tableofcontents          % Inhaltsverzeichnis ausgeben
\clearpage
\pagenumbering{arabic}

\section{Einleitung}
\vspace{0.2cm}
Dieses Dokument dient dazu, einen allgemeinen Überblick darüber zu geben, was wir, das BalancedBanana-Team, in der Implementierungsphase gemacht haben. \par
\vspace{5mm}

Das erste Abschnitt beschreibt die Probleme, die wir während dieser Phase hatten, und die Änderungen, die wir gegenüber dem ursprünglichen Pflichtenheft und Entwurf vorgenommen haben. Dieser Abschnitt ist in mehrere Unterabschnitte unterteilt, die jeweils ein Modul unseres Programms beschreiben.\par 
\vspace{5mm}

Das nächste Abschnitt listet kurz die verschiedenen externen Bibliotheken und Programme auf, die wir verwendet haben.\par 
\vspace{5mm}

Danach folgt ein Abschnitt über allgemeine Statistiken über unser Programm einschließlich der Testabdeckung.\par
\vspace{5mm}

Das letzte Abschnitt gibt einen allgemeinen Überblick über unseren ursprünglichen Zeitplan und den tatsächlichen Zeitplan für das Projekt.

\clearpage
\section{Probleme, Änderungen am Pflichtenheft und Entwurf}
\subsection{Allgemein}
\vspace{0.2cm}
Viele der Probleme zu Beginn der Phase standen im Zusammenhang mit der Einrichtung unserer einzelnen Systeme. Die Konfiguration der verschiedenen Entwicklungsumgebungen war zu Beginn sehr zeitaufwändig und die Fehlerbehebung war manchmal schwierig, da jede Person eine andere Einrichtung hatte. Es wäre viel einfacher gewesen, wenn wir alle die exakt gleiche Entwicklungsumgebung auf dem gleichen Betriebssystem verwendet hätten.
Durch teilweiße fehlender Synchronisierung mit der master Branch, entstanden vermeidbare Merge Konflikte.
Wegen vieler Zyklen zwischen den Paketen, wurde die Task klasse in communication verschoben.
Jedoch wurden weitere größere Paketänderungen in die nächste Phase verschoben, um die Implementierung und ausstehende Merges nicht verkomplizieren.
\clearpage
\subsection{Änderungen am Pflichtenheft}
\subsubsection{CommandLineInterface}
\vspace{0.2cm}
Befehle wurden restrukturiert, benutzen jetzt unterbefehle (wie bei git)
Verkürzung der Abkürzungen: -minc -> -c, -maxc -> C, minr -> -r, -maxr -> -R, -ri -> I, -ai -> i, weil CLI11 für Options Abkürzungen einen einzelnen Buchstaben erfordert.

\subsection{Änderungen am Entwurf}
\subsubsection{Datenbank}
\vspace{0.2cm}
Viele der früheren Probleme mit der Implementierung der Datenbank waren mit dem Setup und meiner Unkenntnis von MySQL verbunden. Im Allgemeinen gab es jedoch keine großen Probleme mit der Implementierung des Datenbankteils unseres Programms. Ich habe die Zeit, die es nicht nur für die Implementierung der Datenbank, sondern auch für das Testen der Datenbank brauchte, grob unterschätzt. Das Testen der Datenbank nahm aufgrund der Natur relationaler Datenbanken enorm viel Zeit in Anspruch. Die aktuelle Implementierung folgt im Allgemeinen dem ursprünglichen Entwurf, es gab jedoch einige Änderungen.\par 
\vspace{5mm}
Eine große Änderung war die Aufspaltung der Gateway-Klasse in drei kleinere Klassen (Job-, Worker- und UserGateway). Damit sollte vermieden werden, dass es eine große Götterklasse gibt, die schwer zu warten und zu testen ist. Alle Gateway-Klassen erben von IGateway. Worker- und UserGateway haben sich im Vergleich zum ursprünglichen Entwurf nicht viel geändert, jedoch wurden neue Methoden zu JobGateway hinzugefügt, die von anderen Teilen unseres Programms benötigt wurden.\par
\vspace{5mm}
Eine weitere Änderung war das Hinzufügen eines Caches zum Repository. Der Cache ermöglicht es unserem Programm, verschiedene Objekte zu speichern, so dass andere Teile unseres Programms die Datenbank nicht ständig abfragen müssen.\par
\vspace{5mm}
Eine kleinere Änderung bestand darin, dass die IGateway-Klasse die Verbindung zur Datenbank aufbauen konnte, anstatt sie von der Gateway-Klasse erstellen und ein QSqlDatabase-Objekt als Klassenmitglied speichern zu lassen. Diese Änderung wurde aufgrund einer genaueren Untersuchung der Dokumentation vorgenommen.

\subsubsection{Config Files}
\vspace{0.2cm}
Da es sich bei den Config Files um sehr einfache Container handelt, die keine große Funktion haben, musste hier nichts groß geändert werden. Lediglich die SchedulerConfig Klasse wurde in ApplicationConfig umbenannt, da sie auch in der Worker-Anwendung eingesetzt werden kann. 

\subsubsection{Communication}
\vspace{0.2cm}
Das Communication Paket war im Entwurf noch sehr abstrakt gehalten. Daher ist während der Implementierung vieles dazugekommen, besonders neue Message-Typen. Mittlerweile gibt es:
\begin{itemize}
	\item AuthResultMessage
	\item ClientAuthMessage
	\item HardwareDetailMessage
	\item PublicKeyAuthMessage
	\item RespondToClientMessage
	\item TaskMessage
	\item TaskResponseMessage
	\item WorkerAuthMessage
	\item WorkerLoadRequestMessage
	\item WorkerLoadResponseMessage
\end{itemize}
Eine weitere große Änderung ist die Aufteilung des Communicator in den Communicator und den CommunicatorListener. Letzterer übernimmt jetzt das Warten an einem Port auf Verbindungsanfragen, während der Communicator nur noch für das Empfangen und Senden von Messages verantwortlich ist. Für die Aufteilung haben wir und deshalb entschieden, da auch in allen Netzwerk-Bibliotheken zwischen Socket und Server unterschieden wird. Für das Verarbeiten von Messages sind weiterhin die Unterklassen des MessageProcessor verantwortlich. Diese befinden sich jetzt in dem entsprechenden Anwendungspaket (Scheduler, Worker oder Client) und nicht länger im Communication Paket, da dieses nur als Grundlage für die Kommunikation dienen soll, nicht jedoch die gesamte Verarbeitung für alle drei Anwendungen.

\clearpage
\section{Externe Abhängigkeiten}
\begin{itemize}
	\item [\textbf{QT}] \hspace{10mm} Bibliothek für Netzwerkübertragung und Datenbanken. Wird verwendet, um die drei Programme (Client, Scheduler und Worker) miteinander kommunizieren lassen zu können. Wird außerdem dazu verwendet die MySQL Datenbank auf dem Scheduler zu verwalten und von innerhalb des Codes SQL Anfragen zu ermöglichen.

	\item [\textbf{CLI11}]\hspace{10mm} Bibliothek zum einlesen der Befehlszeile. Wird verwendet um die Argumente beim Aufruf der drei Programme (Client, Scheduler und Worker) die der Benutzer angegeben hat einzulesen.

	\item [\textbf{OpenSSL}] \hspace{10mm} Bibliothek zur Verschlüsselung von Daten. Wird dazu verwendet einen Benutzer mithilfe eines Public-Private-Key Verfahrens mit dem Scheduler zu verbinden und zu identifizieren, wenn sich der selbe Benutzer erneut im System anmeldet.

	\item [\textbf{GTest}] \hspace{0.5cm} Bibliothek zum Testen von C und C++ Code. Wird verwendet, um die einzelnen Bestandteile der Programme zu testen.

	\item [\textbf{SimpleHTTPServer}] \hspace{3cm} Selbstentwickelte Bibliothek, die das Aufsetzten und verwalten eines HTTP Servers ermöglicht. Wird verwendet, um den HTTP Server zu verwalten, der auf HTTP Anfragen bezüglich des Systemstatus (z.B. Auslastung der Arbeiter Rechner, ...) antwortet.

	\item [\textbf{Docker}]\hspace{10mm} Bibliothek zum kapseln von Prozessen in Containern. Wird dazu verwendet, die Benutzer Aufträge voneinender zu trennen und die gesetzten Grenzen für die Hardwareressourcen einzubehalten.

	\item [\textbf{Cereal}]\hspace{1cm} Bibliothek zur Serialisierung.

	\item [\textbf{criu}] \hspace{1cm}Bibliothek zur Checkpoint-/Wiederherstellungsfunktionalität 

	\item [\textbf{python}]\hspace{1cm} Programmiersprache für Scripts. Wird verwendet, um gewisse Tests auszuführen, die Netzwerkfunktionalität erfordern.

\end{itemize}
\section{Statistik}
\subsection{Allgemein}
\subsection{Testfälle}


\clearpage

\section{Implementierungsplan}
\subsection{Geplant}


\subsection{Realität}

\clearpage

\end{document}
