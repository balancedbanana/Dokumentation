\documentclass[parskip=full]{scrartcl}
\usepackage[utf8]{inputenc}
\usepackage[T1]{fontenc}
\usepackage[german]{babel}
\usepackage{hyperref}
\hypersetup{pdftitle={Balanced Banana},bookmarks=true,}
\usepackage{graphicx}
\usepackage{csquotes}
\usepackage[nonumberlist]{glossaries}
\usepackage{enumitem}
\usepackage{verbatim}

\makeatletter
\newenvironment{mycode}
 {\def\@xobeysp{\ }\verbatim\rightskip=0pt plus 6em\relax}
 {\endverbatim}
\makeatother

\setitemize{align=parleft, labelsep=0.5cm}

\makenoidxglossaries

\newglossaryentry{CLI}
{
	name=CLI,
	description={Commandline Interface},
}
\newglossaryentry{Client}{name={Client},description={Programm auf dem Computer eines Benutzers um mit dem Server zu  kommunizieren}}
\newglossaryentry{Configfile}{name={Configfile},description={Eine Datei die bestimmte Einstellungen speichert}}

\title{Balanced Banana}
\subtitle{Distributed Task Scheduling}
\author{Niklas Lorenz \and Thomas Häuselmann \and Rakan Zeid Al Masri \and Christopher Lukas Homberger \and Jonas Seiler}

\begin{document}

\maketitle

\section{Einleitung}

\section{Zielbestimmung}

\section{Produkteinsatz}

\section{Produktumgebung}

\section{Funktionale Anforderungen}
\subsection{Liste der funktionalen Anforderungen}

\subsubsection{Kernanforderungen} %Name ändern pls

\begin{itemize}[nosep]
\leftskip=0.5cm

\begin{comment}

%Format einer funktionalen Anforderung:
\begin{minipage}[t]{\linewidth}
\item[FA00] \textbf{<Titel>}
\subitem \textbf{Erklärung} <In ca. 3 Zeilen eine grobe Beschreibung geben>
\subitem \textbf{Wichtigkeit} <entweder Kern-Funktionalität oder Optionale-Funktionalität>
\subitem \textbf{Vorraussetzungen} <Wann ist diese Funktion nutzbar?> <dieser Punkt kann weggelassen werden>
\subitem \textbf{Nachbedingung} <Dieser Punkt kann weggelassen werden>
\subsubitem \textbf{Erfolg} <Was geschieht wenn diese Funktion erfolgreich ausgeführt wurde>
\subsubitem \textbf{Misserfolg} <Was geschieht wenn diese Funktion nicht ausgeführt werden kann>
\subitem \textbf{Auslöser} <Wie wird diese Funktion gestartet> <Dieser Punkt kann weggelassen werden>
\subitem \textbf{Details} <Ausführliche Beschreibung dieser funktionalen Anforderung>
\end{minipage}
\pagebreak
%Ende der Vorlage

\end{comment}

\item[FA10]	\gls{Client} verbindet sich beim Starten mit dem Server
\item[FA20] Benutzer kann eine Aufgabe über \gls{Client} einreihen
\item[FA30] Benutzer kann Parameter übergeben %evtl auf FA20 verweisen
\item[FA31]	\gls{Client} speichert voreingestellte Parameter in \gls{Configfile}
\item[FA32]	Benutzer kann Parameter über \gls{Configfile} übergeben
\item[FA33] Benutzer kann Priorität einer Aufgabe festlegen %Glossar Priorität???
\item[FA34] Benutzer kann minimale und maximale Anzahl genutzer Kerne festlegen %Glossar Kerne
\item[FA35] Benutzer kann maximal nutzbaren RAM festlegen %Glossar RAM
\item[FA40] Anfragen nach Status
\item[FA50] Benutzer bekommt Benachrichtigung über erledigte Aufgabe 
\end{itemize}

\subsubsection{optionale Anforderungen}
\begin{itemize}[nosep]
\leftskip=0.5cm
\item[OFA01] Benutzer kann eine geschätzte Restzeit einer Aufgabe sehen	
\end{itemize}

\section{Produktdaten}
\begin{itemize}[nosep]
\leftskip=0.5cm
\item[PD10]
\end{itemize}

\section{Nichtfunktionale Anforderungen}
\begin{itemize}[nosep]
\leftskip=0.5cm
\item[NF10]
\end{itemize}

\section{Systemmodelle}

\subsection{Szenarien}

\subsection{Anwendungsfälle}

\section{Benutzer Oberfläche}
\subsection{\glspl{CLI}}
\begin{mycode}
bb [--block|-b] [<--priority|-p> <Integer>] [<--docker-file|-df> <path to dockerfile>] [<--max-cpu-count|-Mc> <Integer>] [<--min-cpu-count|-mc> <Integer>] [<--max-ram|-Mr> <Integer>] [<--min-ram|-mr> <Integer>] <program> [args]+
\end{mycode}

\subsubsection{\texttt{-{}-block}}
Kehrt erst nach der Ausführung der Aufgabe zum Aufrufer zurück. Nützlich um voneinander abhängige Aufgaben, in einem Konsolen Skript, nacheinander auszuführen.

\printnoidxglossaries

\end{document}
