\documentclass[a4paper,12pt]{article}
\usepackage{amssymb} % needed for math
\usepackage{amsmath} % needed for math
\usepackage[utf8]{inputenc} % this is needed for german umlauts
\usepackage[ngerman]{babel} % this is needed for german umlauts
\usepackage[T1]{fontenc}    % this is needed for correct output of umlauts in pdf
\usepackage[margin=2.5cm]{geometry} %layout
\usepackage{booktabs}
\usepackage{hyperref}
\hypersetup{pdftitle={Balanced Banana},bookmarks=true,}
\usepackage{graphicx}
\usepackage{csquotes}
\usepackage[nonumberlist]{glossaries}
\usepackage{enumitem}
\usepackage{verbatim}
\usepackage{indentfirst} % Adds indent for the first paragraph after a {/section}


\deftranslation[to=ngerman]{Glossary}{\section{Stichwortverzeichnis}}

\makeatletter
\newenvironment{mycode}
 {\def\@xobeysp{\ }\verbatim\rightskip=0pt plus 6em\relax}
 {\endverbatim}
\makeatother

\setitemize{align=parleft, labelsep=0.5cm}

\makenoidxglossaries

\newglossaryentry{CLI}
{
	name=CLI,
	description={Befehlszeile, engl. Commandline Interface},
}

\newglossaryentry{CPU}
{
	name=CPU,
	description={Central Processing Unit, Kern jedes Rechners um Anwendungen auszuführen},
}

\newglossaryentry{Daemon}
{
	name=Daemon,
	description={Ein Dienst der auf Anfragen reagiert und antwortet},
}
\newglossaryentry{Client}
{
	name={Client},
	description={Programm auf dem Computer eines Benutzers um mit dem Server zu  kommunizieren}
}
\newglossaryentry{Configfile}
{
	name={Configfile},
	description={Eine Datei die bestimmte Einstellungen speichert}
}

\newglossaryentry{Server}
{
	name={Server},
	description={Programm auf dem Server, der die Benutzer (Außenwelt) mit den Arbeitern (Privates Netzwerk) verbindet.}
}

\newglossaryentry{Benutzer}
{
	name={Benutzer},
	description={Eine Person, die dazu in der Lage ist, Befehle auszuführen.}
}

\newglossaryentry{Prioritaet}
{
	name={Priorität},
	plural={Prioritäten},
	description={Ein diskretes Maß für Wichtigkeit bzw. Relevanz. Meist eine ganze Zahl, wobei gewisse Werte auch durch vorher definierte Wörter bezeichnet werden können.}
}

\newglossaryentry{Multicast}
{
	name={Multicast},
	description={Eine besondere Form von Netzwerk Packet, welches an alle Adressen des Netzwerkes gesendet wird.}
}

\newglossaryentry{Worker}
{
	name={Worker},
	description={Programm auf den Rechnern, welche dafür zuständig sind, die Aufgaben auszuführen.}
}

\newglossaryentry{Administrator}
{
	name={Administrator},
	description={Eine Person, die dazu privilegiert ist, das Gesamtsystem zu verwalten und über uneingeschränkten Zugriff auf Ein- und Ausgabedateien verfügt.}
}

\title{Balanced Banana}
\author{Niklas Lorenz \and Thomas Häuselmann \and Rakan Zeid Al Masri \and Christopher Lukas Homberger \and Jonas Seiler}


%%%%%%%%%%%%%%%%%%%%%%%%%%%%%%%%%%%%%%%%%%%%%%%%%%%%%%%%%%%%%%%%%%%%%%
% Create a shorter version for tables. DO NOT CHANGE               	 %
%%%%%%%%%%%%%%%%%%%%%%%%%%%%%%%%%%%%%%%%%%%%%%%%%%%%%%%%%%%%%%%%%%%%%%
\newcommand\addrow[2]{#1 &#2\\ }

\newcommand\addheading[2]{#1 &#2\\ \hline}
\newcommand\tabularhead{\begin{tabular}{lp{13cm}}
\hline
	}

\newcommand\addmulrow[2]{ \begin{minipage}[t][][t]{2.5cm}#1\end{minipage}%
   &\begin{minipage}[t][][t]{8cm}
    \begin{enumerate} #2   \end{enumerate}
    \end{minipage}\\ }

\newenvironment{usecase}{\tabularhead}
{\hline\end{tabular}}

\usepackage{microtype}

\begin{document}
\pagenumbering{roman}
\begin{titlepage}
    \begin{center}
    
     \vspace*{0.8cm}
 
        \includegraphics[width=0.5\textwidth]{balancedbanana}
        \vspace*{1cm}
 
        \Huge
        \textbf{Balanced Banana}
 
        \vspace{0.5cm}
        \LARGE
        A Distributed Task Scheduling System
        
        \vspace{0.5 cm}
        \LARGE
        Implementierung
 
        \vspace{1.5cm}

        \large
        \textbf{Niklas Lorenz, Thomas Häuselmann, Rakan Zeid Al Masri, Christopher Lukas Homberger und Jonas Seiler}
 
        \vspace*{0.5cm}

        \textbf{\today}
 
       
        
 
    \end{center}
\end{titlepage}         % Deckblatt.tex laden und einfügen
\setcounter{page}{2}
\tableofcontents          % Inhaltsverzeichnis ausgeben
\clearpage
\pagenumbering{arabic}

\section{Einleitung}
\vspace*{1cm}

Aufgabenverteilung ist ein in vielen Unternehmen übliches Problem. Wenn ein Unternehmen größer wird, 
so werden auch die verfügbaren Rechenressourcen und die darauf ausgeführten Aufgaben größer und komplexer sein. Es wird immer schwieriger, die genannten Ressourcen effizient und gerecht auf die verschiedenen Teams, Mitarbeiter und Aufgaben aufzuteilen. \\

Viele Lösungen gibt es bereits auf dem Markt, sie sind aber zu komplex und nicht leicht erweiterbar. Balanced Banana löst diese Probleme in Form eines kompakten, einfach zu bedienenden und skalierbaren Programms. \\

Mit unserem Programm kann der Benutzer seine Aufgabe einfach von der Kommandozeile absetzen. Darüber hinaus ist der Benutzer durch die Verwendung von Parametern in der Lage, seinem Programm zusätzliche Einschränkungen und/oder Bedingungen hinzuzufügen. Mit Hilfe intelligenter Algorithmen ist unser Programm in der Lage, die verfügbaren Rechenressourcen für Aufgaben effizient zu verteilen, basierend auf Größe, \gls{Prioritaet}, Einschränkungen und Bedingungen.

\section{Zielbestimmung}
\subsection{Musskriterien}
\subsection{Wunschkriterien}
\subsection{Abgrenzungskriterien}
% Was will ich bewusst nicht umsetzen?
% Was soll es nicht sein?

% Ist das nicht ne Wiederholung von Einleitung + Anforderungen?
% Ich würde das sehr kurz halten, so wie die Einleitung zum Beispiel

\section{Szenarien}

\section{Produkteinsatz}
% Zielgruppe
% Anwendungsbereiche
% Betriebsbedinugen
% Wer? Was? Wozu?

\section{Produktumgebung}
% Unter welcher Software / Hardware läuft es?

\section{Funktionale Anforderungen}

\subsection{Übersicht der Anforderungen}

\subsubsection{Kernanforderungen} %Name ändern pls

\begin{itemize}[nosep]
\leftskip=0.5cm

\item[FA1]	\gls{Client} verbindet sich beim Starten mit dem Server
\item[FA2] Benutzer authentifiziert sich über den \gls{Client} gegenüber dem Server
\item[FA3] Benutzer kann eine Aufgabe über \gls{Client} einreihen
\item[FA4] Benutzer kann Parameter übergeben %evtl auf FA20 verweisen
\item[FA41]	\gls{Client} speichert voreingestellte Parameter in \gls{Configfile}
\item[FA42]	Benutzer kann Parameter über \gls{Configfile} übergeben
\item[FA43] Benutzer kann \gls{Prioritaet} einer Aufgabe festlegen %Glossar \gls{Prioritaet}???
\item[FA44] Benutzer kann minimale und maximale Anzahl genutzer Kerne festlegen %Glossar Kerne
\item[FA45] Benutzer kann maximal nutzbaren RAM festlegen %Glossar RAM
\item[FA46] Benutzer kann das benutzte Betriebssystem festlegen
\item[FA47] Benutzer kann angeben ob die Aufgabe pausierbar ist. %evtl optional
\item[FA48] Benutzer kann angeben ob der \gls{Client} blockieren soll bis die Aufgabe beendet ist. %Blockierbar glosar?
\item[FA49] Benutzer übergibt Pfad zu den für die Aufgabe benötigten Dateien. %Pfad glosar?
\item[FA5] Benutzer kann den Status einer Aufgabe einsehen
\item[FA51] Benutzer kann die verstrichene Zeit einer Aufgabe einsehen %evtl optional
\item[FA6] Benutzer bekommt Benachrichtigung über erledigte Aufgabe
\item[FA7] Server erstellt regelmäßige Sicherungen von laufenden Aufgaben
\item[FA8] Benutzer kann die Ausgabe seiner Aufgabe anfordern
\end{itemize}

% Ist die Aufgabe pausierbar?

% Wie kann die Aufgabe abgebrochen oder pausiert werden (Einen Befehl spezifizieren)

% Soll der Aufrufer im Terminal blockiert werden

\subsubsection{optionale Anforderungen}
\begin{itemize}[nosep]
\leftskip=0.5cm
\item[OFA1] Benutzer kann eine geschätzte Restzeit einer Aufgabe sehen	
\item[OFA2] Server stoppt Aufgaben die zu lange dauern %Dauer einfügen
\item[OFA3] Benutzer kann eine manuelle Stoppung seiner Aufgabe anfordern
\item[OFA4] Benutzer kann eine manuelle Sicherung seiner Aufgabe anfordern
\end{itemize}
% Es soll einen Webserver als Mittelmann geben

% Es soll möglich sein, Backups von laufenden Aufgaben zu erstellen

% Aufgaben, die nach 2 Tagen arbeitszeit noch nicht beenden, sollen unterbrochen werden und dem Auftraggeber soll eine Mitteilung geschickt werden

% Die Anwendung soll über verschiedene Betriebsmodi verfügen (Client, Server, Worker, Admin?)

% Anforderung eines manuellen Backups / Pausierung / Abbruch

% An wen soll die Rückmeldung erfolgen (EMail oder Nutzerkonto)

% Am Ende des unseres Befehls folgt der Befehl mit dem die Aufgabe gestartet werden kann
\subsubsection{Funktionale Anforderungen}

\begin{comment}

%Format einer funktionalen Anforderung:
\begin{minipage}[t]{\linewidth}
\item[FA00] \textbf{<Titel>}
\subitem \textbf{Erklärung} <In ca. 3 Zeilen eine grobe Beschreibung geben>
\subitem \textbf{Wichtigkeit} <entweder Kern-Funktionalität oder Optionale-Funktionalität> %würde ich weglassen
\subitem \textbf{Vorraussetzungen} <Wann ist diese Funktion nutzbar?> <dieser Punkt kann weggelassen werden> %absolut nicht, das ist wichtig für testfälle
\subitem \textbf{Nachbedingung} <Dieser Punkt kann weggelassen werden>
\subsubitem \textbf{Erfolg} <Was geschieht wenn diese Funktion erfolgreich ausgeführt wurde>
\subsubitem \textbf{Misserfolg} <Was geschieht wenn diese Funktion nicht ausgeführt werden kann>
\subitem \textbf{Auslöser} <Wie wird diese Funktion gestartet> <Dieser Punkt kann weggelassen werden>
\subitem \textbf{Details} <Ausführliche Beschreibung dieser funktionalen Anforderung>
\end{minipage}
\pagebreak
%Ende der Vorlage

\end{comment}

\begin{itemize}[nosep]
\leftskip=0.5cm
\begin{comment}
\item[FA1] \textbf{\gls{Client} verbindet sich beim Start mit dem Server.}
\begin{itemize}[nosep]
\item \textbf{Erklärung:} Beim Starten der Client-Anwendung versucht diese sich automatisch mit der Serveranwendung zu verbinden.
\item \textbf{Vorraussetzungen:} Keine.
\item \textbf{Erfolg:} Die Client-Anwendung konnte sich mit der Serveranwendung verbinden. Eine entsprechende Meldung wird ausgegeben.
\item \textbf{Misserfolg:} Die Client-Anwendung konnte sich nicht mit der Serveranwendung verbinden. Eine entsprechende Fehlermeldung wird ausgegeben.
\item \textbf{Auslöser:} Die Client-Anwendung wird gestartet.
\item \textbf{Details:}
\begin{itemize}[nosep]
	\item Wenn der Benutzer die Client-Anwendung startet, versucht diese die zugehörige Serveranwendung zu finden und sich mit dieser zu verbinden.
\end{itemize}
\end{itemize}
\end{comment}
%\item[FA10]	\gls{Client} verbindet sich beim Starten mit dem Server
\begin{minipage}[t]{\linewidth}
\item[FA10] \textbf{Automatisierter Verbindungsaufbau}
\subitem \textbf{Erklärung} Der \gls{Client} soll ohne explizite Aufforderung durch den \gls{Benutzer} mit dem \gls{Server} eine Netzwerk Verbindung zum Übermitteln der Daten aufbauen.
\subitem \textbf{Wichtigkeit} Kernfunktionalität
\subitem \textbf{Voraussetzung(en)} Ein \gls{Server} muss erreichbar sein.
\subitem \textbf{Nachbedingung(en)}
\subsubitem \textbf{Erfolg} Der \gls{Client} ist in der Lage jeden Befehlsaufruf, ohne den \gls{Benutzer} darüber zu informieren, an den \gls{Server} zu übermitteln.
\subsubitem \textbf{Misserfolg} Eine Fehlernachricht wird ausgegeben, die den \gls{Benutzer} darüber in Kenntnis setzt, dass keine Netzwerk Verbindung mit dem \gls{Server} aufgebaut werden konnte.
\subitem \textbf{Auslöser} Jeder Befehlsaufruf des \gls{Client} löst einen automatisierten Verbindungsaufbau aus.
\subitem \textbf{Details} Zur einfachen Verwendung eines \gls{Client} soll sich dieser selbstständig mit einem \gls{Server} verbinden.\newline
Die Verbindung wird zum Start jeder Übermittlung aufgebaut und nach Ende der Übermittlung geschlossen.\newline
Die Verbindung dient einzig dem Zweck, den \gls{Server} über einen Befehlsaufruf  in Kenntnis zu setzen.\newline
Ablauf:\newline
Schritt 1: Der \gls{Client} sendet eine \gls{Multicast} Nachricht an das Netzwerk.\newline
Schritt 2: Der \gls{Server} beantwortet die Nachricht und setzt so den \gls{Client} über seine Existenz und Netzwerk-Adresse in Kenntnis.\newline
Schritt 3: Der \gls{Client} übermittelt den Befehlsaufruf an den Server.\newline
Schritt 4: Der \gls{Server} sendet eine Eingangsbestätigung an den \gls{Client}.\newline
Schritt 5: Da nun keine Nachrichten mehr übermittelt werden müssen, wird die Verbindung geschlossen.
\end{minipage}
\pagebreak

\begin{minipage}[t]{\linewidth}
\item[FA11] \textbf{Gedächtnis der vorherigen Verbindung}
\subitem \textbf{Erklärung} Der \gls{Client} soll sich Informationen über eine erfolgreiche Verbindung zu einem \gls{Server} merken.
\subitem \textbf{Wichtigkeit} Optional
\subitem \textbf{Voraussetzung(en)} Diese Funktionalität wird stets in Verbindung mit Funktion FA10 verwendet.\newline
Funktion FA10 konnte erfolgreich abgeschlossen werden.
\subitem \textbf{Nachbedingung(en)}
\subsubitem \textbf{Erfolg} Der \gls{Client} ist bei dem nächsten Verbindungsaufbau in der Lage, den \gls{Server} direkt anzusprechen.
\subsubitem \textbf{Misserfolg} Der \gls{Client} hat keine Informationen über einen existierenden \gls{Server}.
\subitem \textbf{Auslöser} Funktion FA10 wurde erfolgreich ausgeführt.
\subitem \textbf{Details} Der Verbindungsaufbau zwischen \gls{Client} und \gls{Server} mithilfe einer \gls{Multicast} Nachricht sollte vermieden werden. Daher soll sich der \gls{Client} den letzten \gls{Server} mit dem er sich erfolgreich verbunden hat merken. Somit kann der \gls{Client} bei dem nächsten Verbindungsaufbau die \gls{Multicast} Nachricht vermeiden, indem er sich direkt mit dem \gls{Server} in Verbindung setzt.\newline
Ist der hinterlegte \gls{Server} nicht mehr erreichbar, so wird er vom \gls{Client} vergessen.\newline
Der Ablauf von Funktionalität FA10 ändert sich wie folgt:\newline
Schritt 1: Der \gls{Client} versucht den vorgemerkten \gls{Server} zu erreichen. Ist dies nicht möglich, vergisst er diesen und sendet eine \gls{Multicast} Nachricht an das Netzwerk.\newline
Schritt 2-5: Unverändert
Schritt 6: Der \gls{Client} speichert sich die Netzwerkadresse des \gls{Server} ab.
\end{minipage}
\pagebreak

%\item[FA20] Benutzer kann eine Aufgabe über \gls{Client} einreihen
\begin{minipage}[t]{\linewidth}
\item[FA20] \textbf{Warteschlange}
\subitem \textbf{Erklärung} Sollte zum Zeitpunkt des Eingangs einer neuen Aufgabe kein \gls{Worker} mit hinreichend Arbeitskapazität verfügbar sein, so soll die Aufgabe nicht abgelehnt oder verloren gehen, sondern in eine Warteschlange eingereiht werden. Zu einem späteren Zeitpunkt kann die Aufgabe dann ausgeführt werden.
\subitem \textbf{Wichtigkeit} Kernfunktionalität
\subitem \textbf{Voraussetzung(en)} Diese Funktion ist auf jedem \gls{Server} verfügbar.
\subitem \textbf{Nachbedingung(en)}
\subsubitem \textbf{Erfolg} Die Aufgabe ist in einer Warteschlange eingereiht.
\subsubitem \textbf{Misserfolg} Der Auftraggeber (\gls{Client}) wird darüber in Kenntnis gesetzt, dass seine Aufgabe nicht bearbeitet wird.\newline
Ein \gls{Administrator} wird davon in Kenntnis gesetzt, dass ein Fehler beim Aufnehmen einer Aufgabe in die Warteschlange aufgetreten ist.
\subitem \textbf{Auslöser} Der \gls{Server} erhält eine Aufgabe zu einem Zeitpunkt, an dem kein geeigneter \gls{Worker} verfügbar ist.
\subitem \textbf{Details} Alle von einem \gls{Benutzer} in Auftrag gegebenen Aufgaben sollen angenommen und ausgeführt werden. Dies ist jedoch aufgrund beschränkter Rechenkapazität nicht immer sofort möglich. Daher sollen Aufgaben, die nicht sofort bearbeitet werden können, vorübergehend vom \gls{Server} in einer Warteschlange gehalten werden. Nachdem ausreichend Rechenkapazität frei geworden ist, soll die Warteschlange wieder geleert werden.
\end{minipage}
\pagebreak

%\item[FA30] Benutzer kann Parameter übergeben %evtl auf FA20 verweisen
\begin{minipage}[t]{\linewidth}
\item[FA30] \textbf{Befehlsparameter}
\subitem \textbf{Erklärung} Dem Programm können per Befehlszeile bestimmte Optionen bzw. Informationen bezüglich der Ausführung des gegebenen Befehls bereitgestellt werde.
\subitem \textbf{Wichtigkeit} Kernfunktionalität
\subitem \textbf{Voraussetzung(en)} Jeder Befehl kann mit einem gewissen Satz von Parametern aufgerufen werden.\newline
Die Parameter sind für den angegebenen Befehl zulässig.\newline
Es sind keine Parameter angegeben, die sich gegenseitig ausschließen.\newline
Jeder Parameter wird mit einem zulässigen Eingabewert angegeben.
\subitem \textbf{Nachbedingung(en)}
\subsubitem \textbf{Erfolg} Der Befehl wird unter den angegebenen Parametern ausgeführt.
\subsubitem \textbf{Misserfolg} Dem Benutzer wird mitgeteilt, welcher Parameter unzulässig angegeben wurde.\newline
Der Befehl wird nicht ausgeführt und nicht an den \gls{Server} weitergeleitet bzw. von diesem ignoriert.
\subitem \textbf{Auslöser} Angabe des Parameters auf der Kommandozeile.
\subitem \textbf{Details} Damit das Programm mehr als eine Funktion erfüllen kann oder eine der angebotenen Funktionen mit nicht-standardmäßigen Werten ausführen kann, muss dem Programm mitgeteilt werden, welche Funktion mit welchen Werten gewünscht ist. Hierzu werden Befehlsparameter verwendet (für genauere Angaben zu allen Parametern siehe nachfolgende funktionale Anforderungen sowie Absatz 10 Benutzeroberfläche).\newline
Jeder Parameter kann auf der Kommandozeile durch die Syntax "--<Name>" oder "-<Kürzel>", gefolgt von dem gewünschten Wert, verwendet werden.\newline
Ein Parameter hat entweder durch Angabe eines Wertes oder durch Angabe des Parameters Einfluss auf die Ausführung des Befehls (abhängig vom konkreten Parameter).\newline
Jeder Parameter hat einen eindeutig definierten Standardwert.\newline
Es ist möglich, dass zwei Parameter nicht in dem selben Befehl verwendet werden können und dürfen. Sollte eine solche Kollision dennoch vorhanden sein, schlägt die Ausführung des Befehls fehl.
\end{minipage}
\pagebreak

%\item[FA31]	\gls{Client} speichert voreingestellte Parameter in \gls{Configfile}
\begin{minipage}[t]{\linewidth}
\item[FA31] \textbf{Global definierte Standardwerte}
\subitem \textbf{Erklärung} Jeder Parameter hat einen global definierten Standardwert. Dieser wird verwendet, wenn der \gls{Benutzer} keinen anderen Wert angibt. Ein globaler Standardwert kann nie einen Fehler auslösen.
\subitem \textbf{Wichtigkeit} Kernfunktionalität
\subitem \textbf{Voraussetzung(en)} Dem Parameter kann ein Wert zugewiesen werden.
\subitem \textbf{Nachbedingung(en)} 
\subsubitem \textbf{Erfolg} Der Parameter wird mit einem unkritischen Wert aufgefüllt.\newline
Der Befehl wird trotz unvollständiger Parameter ausgeführt.
\subitem \textbf{Auslöser} Der Benutzer hat keinen eigenen Parameterwert angegeben.
\subitem \textbf{Details} Parameter, bei denen die reine Angabe auf der Kommandozeile nicht Aussagekräftig ist (Der Parameter erfordert z.B. einen numerischen Wert) müssen immer mit einem Wert ausgefüllt werden.\newline
Oftmals ist es lästig einen Parameter explizit anzugeben, da bis auf Randfälle immer der gleiche Wert verwendet wird. In solch einem Fall soll es möglich sein, den Parameter implizit anzugeben.\newline
Sollte ein \gls{Benutzer} vergessen, einen Parameter anzugeben, so soll der Befehl dennoch mit einem stets unkritischen Wert ausgeführt werden.\newline
Sollte ein \gls{Benutzer} nicht darüber im klaren sein, wozu ein bestimmter Parameter dient, soll dem \gls{Benutzer} ein unkritischer Wert vorgeschlagen werden.\newline
Die Standardwerte sind in einer Konfigurationsdatei abgespeichert.
\end{minipage}
\pagebreak

%\item[FA32]	Benutzer kann Parameter über \gls{Configfile} übergeben
\begin{minipage}[t]{\linewidth}
\item[FA32] \textbf{Benutzerdefinierte Standardwerte}
\subitem \textbf{Erklärung} Jeder Parameter hat einen eindeutig definierten Standardwert, der bei Nichtangabe des Parameters auf der Kommandozeile verwendet wird. Der \gls{Benutzer} kann lokal eine eigene Standardbelegung definieren.
\subitem \textbf{Wichtigkeit} Optional
\subitem \textbf{Voraussetzung(en)} Die lokal definierte Standardbelegung darf keine unzulässige Belegung enthalten.\newline
Der Benutzer hat keinen eigenen Parameterwert angegeben.
\subitem \textbf{Nachbedingung(en)}
\subsubitem \textbf{Erfolg} Bei Nichtangabe eines Parameters auf der Kommandozeile werden die von dem \gls{Benutzer} angegebenen Standardwerte vor den globalen Standardwerten eingesetzt.
\subitem \textbf{Auslöser} Der \gls{Benutzer} hat den Parameter nicht auf der Kommandozeile angegeben.
\subitem \textbf{Details} Ein \gls{Benutzer} verwendet möglicherweise bei jedem Befehlsaufruf immer identische Werte für manche Parameter. Sollten diese Werte nicht mit den global definierten Standardwerten übereinstimmen, kann der \gls{Benutzer} individuelle Standardwerte angeben. Somit muss der \gls{Benutzer} nicht bei jedem Aufruf den Parameter explizit angeben.\newline
Die benutzerdefinierten Standardwerte sind in einer ausgezeichneten Konfigurationsdatei abgespeichert.
\end{minipage}
\pagebreak

%\item[FA33] Benutzer kann \gls{Prioritaet} einer Aufgabe festlegen %Glossar \gls{Prioritaet}???
\begin{minipage}[t]{\linewidth}
\item[FA33] \textbf{\gls{Prioritaet} (Parameter)}
\subitem \textbf{Erklärung} Beim Einreichen einer neuen Aufgabe soll der Auftraggeber (\gls{Benutzer}) in der Lage sein, seine Aufgabe mit einer \gls{Prioritaet} zu versehen.
\subitem \textbf{Wichtigkeit} Optional
\subitem \textbf{Voraussetzung(en)} Der Benutzer muss mit dem Befehl eine zulässige Aufgabe einreichen.
\subitem \textbf{Nachbedingung(en)}
\subsubitem \textbf{Erfolg} Die Aufgabe wird mit der gewünschten \gls{Prioritaet} in die Warteschlange eingereiht oder sofort ausgeführt.
\subitem \textbf{Details} Eine \gls{Prioritaet} stellt eine Ordnung auf allen ausstehenden Aufgaben dar. Das bedeutet, Für je zwei (aktive oder passive) Aufgaben lässt sich bestimmen, welche von beiden zuerst ausgeführt werden soll.\newline
Die \gls{Prioritaet} bestimmt somit die Position einer Aufgabe innerhalb der Warteschlange (FA20).\newline
Tritt der Fall ein, dass zwei Aufgaben die selbe \gls{Prioritaet} haben, beide in der Warteschlange liegen und nur Rechenleistung für eine der Aufgaben verfügbar ist, so ist die Ausführungsreihenfolge der Aufgaben zufällig.\newline
Aufgaben mit hohen \glspl{Prioritaet} können aktive Aufgaben mit stark geringerer \gls{Prioritaet} unterbrechen.\newline
Aufgaben, die längere Zeit in der Warteschlange verbringen, werden in ihrer \gls{Prioritaet} aufgewertet.\newline
Eine aktive Aufgabe wird in ihrer \gls{Prioritaet} aufgewertet.\newline
Für eine \gls{Prioritaet} sind folgende Werte zulässig:\newline
geringste: low\newline
normal: medium\newline
hoch: high\newline
sehr hoch: extreme\newline
höchste: bananas\newline
Der globale Standardwert ist festgelegt auf medium.
\end{minipage}
\pagebreak

%\item[FA34] Benutzer kann minimale und maximale Anzahl genutzer Kerne festlegen %Glossar Kerne
\begin{minipage}[t]{\linewidth}
\item[FA34] \textbf{CPU Kerne (Parameter)}
\subitem \textbf{Erklärung} Erlaubt es dem \gls{Benutzer} eine minimale bzw. eine maximale Anzahl von gewünschten bzw. benötigten CPU Kernen anzugeben.
\subitem \textbf{Wichtigkeit} Optional
\subitem \textbf{Voraussetzung(en)} Der Benutzer muss mit dem Befehl eine zulässige Aufgabe einreichen.
\subitem \textbf{Nachbedingung(en)}
\subsubitem \textbf{Erfolg} Der Aufgabe wird das angegebene Minimum an Prozessorkernen garantiert.\newline
Der Aufgabe werden nicht mehr Prozessorkerne zugewiesen, als durch das Maximum angegeben.\newline
\subitem \textbf{Details} Die Angabe von minimaler Anzahl benötigter Prozessorkernen erlaubt es dem Programm sicherzustellen, dass die Aufgabe nur dann gestartet wird, wenn ihr ausreichend Prozessorkerne zur Verfügung stehen.\newline
Die Angabe einer maximalen Anzahl benötigter Prozessorkerne ermöglicht es dem Programm mehrere Aufgaben parallel an einen \gls{Worker} zu geben. Eine notwendige Bedingung dafür, dass mehrere Aufgaben auf dem selben \gls{Worker} laufen können ist, dass die Summe der Maxima aller Aufgaben unter der Anzahl der auf dem \gls{Worker} verfügbaren Prozessorkerne liegt. Somit blockieren Single-Thread Aufgaben nicht zwangsläufig einen ganzen \gls{Worker}.\newline
Für die Anzahl der Prozessorkerne ist jede natürliche Zahl sowie die Null zulässig.\newline
Der Wert null bedeutet hierbei, dass der Aufgabe eine dynamische Anzahl an Prozessorkernen zugewiesen wird.\newline
Der minimale Wert wird auf der Kommandozeile mit "-min-cpu-count" angegeben.\newline
Der maximale Wert wird auf der Kommandozeile mit "-max-cpu-count" angegeben.\newline
Liegt der Minimalwert über der Zahl der verfügbaren Prozessorkerne, wird die Aufgabe mit der größtmöglichen Anzahl Prozessorkerne ausgeführt.\newline
Der globale Standardwert für die minimale Anzahl an Prozessorkernen ist festgelegt auf 1.\newline
Der globale Standardwert für die maximale Anzahl an Prozessorkernen ist festgelegt auf 0.
\end{minipage}
\pagebreak

%\item[FA35] Benutzer kann maximal nutzbaren RAM festlegen %Glossar RAM
\begin{minipage}[t]{\linewidth}
\item[FA35] \textbf{Größe des benötigten Arbeitsspeichers (Parameter)}
\subitem \textbf{Erklärung} Erlaubt es dem Nutzer eine minimale bzw. maximale Menge Arbeitsspeicher anzufordern.
\subitem \textbf{Wichtigkeit} Optional
\subitem \textbf{Voraussetzung(en)} Der Benutzer muss mit dem Befehl eine zulässige Aufgabe einreichen.
\subitem \textbf{Nachbedingung(en)}
\subsubitem \textbf{Erfolg} Die Aufgabe hat das angegebene Minimum an Arbeitsspeicher zur Verfügung.\newline
Die Aufgabe hat nicht mehr als das angegebene Maximum an Arbeitsspeicher zur Verfügung.
\subitem \textbf{Details} Die Angabe eines Mindestwerts erlaubt es dem Programm sicherzustellen, dass die Aufgabe hinreichend Arbeitsspeicher für ihre Ausführung zur Verfügung hat.\newline
Die Angabe eines Maximalwerts erlaubt es dem Programm, mehrere Aufgaben an den selben \gls{Worker} zu verteilen. Eine notwendige Bedingung dafür, dass mehrere Aufgaben auf dem selben \gls{Worker} laufen können ist, dass die Summe der Maxima aller Aufgaben unterhalb des eingebauten Arbeitsspeichers des \gls{Worker} liegt.\newline
Für die Größe des zugewiesenen Arbeitsspeichers ist jede Natürliche Zahl sowie die Null zulässig. Der Wert wird als Angabe in GiB interpretiert. Für kleinere Mengen kann der Suffix k bzw m für KiB bzw. MiB verwendet werden.\newline
Der Wert Null bedeutet hierbei, dass die Größe des zugewiesenen Arbeitsspeichers dynamisch ist.\newline
Der minimale Wert kann auf der Kommandozeile mit "-min-ram" angegeben werden.\newline
Der maximale Wert kann auf der Kommandozeile mit "-max-ram" angegeben werden.\
Liegt der Minimalwert über dem verfügbaren Arbeitsspeicher, wird der Aufgabe die größtmögliche Menge Arbeitsspeicher zugewiesen.\newline
Der globale Standardwert für die minimale Menge Arbeitsspeicher ist festgelegt auf 1GiB.\newline
Der globale Standardwert für die maximale Menge Arbeitsspeicher ist festgelegt auf 8GiB.
\end{minipage}
% Auf welchem Betriebssystem soll die Aufgabe ausgeführt werden?

%\item[FA40] Anfragen nach Status
\begin{minipage}[t]{\linewidth}
\item[FA40] \textbf{Statusabfrage}
\subitem \textbf{Erklärung} Der Ausführungsstatus der in Auftrag gegebenen Aufgabe(n) kann von dem Auftraggeber abgefragt werden.
\subitem \textbf{Wichtigkeit} Optional
\subitem \textbf{Voraussetzung(en)} Der \gls{Client} muss sich mit dem \gls{Server} verbinden können, der die Aufgabe(n) verteilt hat.
\subitem \textbf{Nachbedingung(en)}
\subsubitem \textbf{Erfolg} Dem \gls{Benutzer} wird angezeigt, in welchem Ausführungsstatus sich die Aufgabe(n) befinden.
\subsubitem \textbf{Misserfolg} Dem \gls{Benutzer} wird mitgeteilt, an welchem Punkt die Abfrage fehlgeschlagen ist.
\subitem \textbf{Auslöser} Der \gls{Benutzer} sendet einen Abfragebefehl ab.
\subitem \textbf{Details} Die Statusabfrage ist eine Variante des Hauptbefehls zum Verteilen von Aufgaben und teilt denselben Befehlsnamen.\newline
Die Statusabfrage ist durch den Parameter "--status" gekennzeichnet.\newline
Die Statusabfrage liefert Informationen zu allen in Auftrag gegebenen Aufgaben des Auftraggebers.\newline
Genauere Informationen können mithilfe weiterer Parameter angefragt werden. (siehe folgende funktionale Anforderungen sowie Abschnitt 10 Benutzeroberfläche)
\end{minipage}
\pagebreak

%\item[FA50] Benutzer bekommt Benachrichtigung über erledigte Aufgabe 
\begin{minipage}[t]{\linewidth}
\item[FA50] \textbf{Rückmeldung}
\subitem \textbf{Erklärung} Nach Abschluss einer Aufgabe soll der Auftraggeber über den Abschluss in Kenntnis gesetzt werden.
\subitem \textbf{Wichtigkeit} Optional
\subitem \textbf{Voraussetzung(en)} Der Auftraggeber hat eine gültige E-Mail Adresse hinterlegt.
\subitem \textbf{Nachbedingung(en)}
\subsubitem \textbf{Erfolg} Der Auftraggeber erhält an der hinterlegten E-Mail Adresse eine Nachricht, die das Bearbeitungsende seiner Aufgabe signalisiert.
\subitem \textbf{Auslöser} Eine Aufgabe beendet ihre Ausführung.
\subitem \textbf{Details} Insofern der Auftraggeber eine gültige E-Mail Adresse hinterlegt hat, wird an diese Adresse eine E-Mail versandt, die folgende Informationen enthält:\newline
1. Mitteilung über den Abschluss der Aufgabe.
2. Der Name der Aufgabe (Name des ausgeführten Programms).
3. Das Ende der Log-Datei der Aufgabe.
\end{minipage}
\pagebreak

\end{itemize}

\section{Produktdaten}
\begin{itemize}[nosep]
\leftskip=0.5cm
% Was soll gespeichert werden?

\begin{comment}

%Format eines Produktdatums:
\begin{minipage}[t]{\linewidth}
\item[PD00] \textbf{}
\subitem \textbf{Erklärung} 
\subitem \textbf{Wichtigkeit} 
\subitem \textbf{Details} 
\end{minipage}
\pagebreak
%Ende der Vorlage

\end{comment}

% Ausgabedaten
% Logdaten
\begin{minipage}[t]{\linewidth}
\item[PD10] \textbf{Ausgabedaten}
\subitem \textbf{Erklärung} Aufgaben erzeugen typischerweise Ausgabedaten im Verlauf ihrer Ausführung.
\subitem \textbf{Wichtigkeit} Nicht optional
\subitem \textbf{Details} Ausgabedaten bezeichnen hier die Ausgaben der Aufgabe auf der Standardausgabe (z.B. std:out oder System.out).\newline
Um eine Abfrage der letzten Ausgaben einer Aufgabe zu ermöglichen, müssen diese Ausgaben dem Programm bekannt sein. Sie sind in einer Log-Datei abgespeichert und können auf Anfrage des Auftragsgebers an einen Client übermittelt werden.
\end{minipage}
\pagebreak

% Allgemein alle Parameter unseres Befehls die für diese Aufgabe angegeben wurden
\begin{minipage}[t]{\linewidth}
\item[PD20] \textbf{Befehlsparameter}
\subitem \textbf{Erklärung} Die Werte der Befehlsparameter auf der Kommandozeile.
\subitem \textbf{Wichtigkeit} Nicht-optional
\subitem \textbf{Details} Alle übergebenen Parameter werden zusammen mit ihren Werten für spätere Verwendung abgespeichert.
\end{minipage}
\pagebreak

% Ausführungsbefehl, Pausierbefehl, Abbruchbefehl
\begin{minipage}[t]{\linewidth}
\item[PD21] \textbf{Aufgabenspezifische Befehle}
\subitem \textbf{Erklärung} Jede Aufgabe wird mit einem von dem Benutzer definierten Befehl gestartet, pausiert und abgebrochen, sofern diese Aktionen verfügbar sind.
\subitem \textbf{Wichtigkeit} Nicht optional
\subitem \textbf{Details} Für jede Aufgabe sind insgesamt drei verschiedene Befehle definierbar:\newline
1. Start-Befehl: Wie wird diese Aufgabe gestartet?\newline
2. Pause-Befehl: Wie wird diese Aufgabe pausiert, wenn möglich?\newline
2,5. Fortsetzen-Befehl: Wie kann die pausierte Aufgabe fortgesetzt werden?\newline
3. Abbruch-Befehl: Wie kann diese Aufgabe abgebrochen werden?\newline
Ist für eine Aufgabe eine dieser Aktionen nicht verfügbar, oder muss für diese Aktion nichts besonderes unternommen werden, werden keine besonderen Befehle angegeben und das Programm verwendet nur seine Standardroutine.
\end{minipage}
\pagebreak

% Mit welcher \gls{Prioritaet} ist die Aufgabe gestartet worden
\begin{minipage}[t]{\linewidth}
\item[PD22] \textbf{Aufgabenpriorität}
\subitem \textbf{Erklärung} Hinsichtlich der Warteschlangenfunktionalität ist es interessant, für jede Aufgabe die angegebene \gls{Prioritaet} zu kennen.
\subitem \textbf{Wichtigkeit} Zusammen mit \gls{Prioritaet} Parameter
\subitem \textbf{Details} Die Position einer Aufgabe in der Warteschlange, sowie das Recht andere Aufgaben zu unterbrechen sind wesentlich von der eigenen \gls{Prioritaet} abhängig. Daher muss diese Information abgespeichert werden (sofern vorhanden).
\end{minipage}
\pagebreak

% wer hat die Aufgabe gestartet
\begin{minipage}[t]{\linewidth}
\item[PD23] \textbf{Auftraggeber Kontaktinformation}
\subitem \textbf{Erklärung} Hält Kontaktinformationen (E-Mail Adresse) des Auftraggebers.
\subitem \textbf{Wichtigkeit} Optional
\subitem \textbf{Details} Um dem Auftraggeber Rückmeldung über abgeschlossene Aufgaben geben zu können, muss eine Kontaktinformation hinterlegt sein.\newline
Es ist vorgesehen, eine E-Mail Adresse als Kontaktinformation anzunehmen. Diese wird beim Einreichen der Aufgabe von dem Auftraggeber angegeben.
\end{minipage}
\pagebreak

% Auf welchem Betriebssystem soll die Aufgabe ausgeführt werden
\begin{minipage}[t]{\linewidth}
\item[PD24] \textbf{Zielbetriebssystem}
\subitem \textbf{Erklärung} Speichert Betriebssystem Vorgaben für Aufgaben, die ein spezielles Betriebssystem erfordern.
\subitem \textbf{Wichtigkeit} Optional
\subitem \textbf{Details} Manche Programme sind Abhängig vom Betriebssystem. Diesen Programmen soll daher ermöglicht werden, auf ihrem gewünschten Zielbetriebssystem zu laufen. Hierzu ist es notwendig, Informationen über besagtes Zielbetriebssystem zu speichern.
\end{minipage}
\pagebreak

% Ist das Programm pausierbar
\begin{minipage}[t]{\linewidth}
\item[PD25] \textbf{Pausierbarkeit}
\subitem \textbf{Erklärung} Hinsichtlich der Warteschlangenfunktionalität ist es notwendig für jede Aufgabe zu wissen, ob diese unterbrochen (pausiert) werden kann.
\subitem \textbf{Wichtigkeit} Nicht optional
\subitem \textbf{Details} Die Information über die Pausierbarkeit einer Aufgabe wird beim Einreichen der Aufgabe als Parameter mitgegeben.\newline
Die Information über die Pausierbarkeit ist ein Wahrheitswert.
\end{minipage}
\pagebreak

%Aufgabenspezifische Statistiken
\begin{minipage}[t]{\linewidth}
\item[PD40] \textbf{Aufgabenspezifische Statistiken}
\subitem \textbf{Erklärung} Das Programm soll in der Lage sein, Kennwerte bezüglich der einzelnen Aufgaben zu erheben.
\subitem \textbf{Wichtigkeit} Optional
\subitem \textbf{Details} Die einzelnen Kennwerte sind im folgenden aufgelistet.\newline
Ein aufgabenspezifischer Kennwert enthält eine Information über eine einzelne Aufgabe.
\end{minipage}
\pagebreak

% Ausführungszeit (insgesamt, aktiv, passiv)
\begin{minipage}[t]{\linewidth}
\item[PD41] \textbf{Arbeitszeiten}
\subitem \textbf{Erklärung} Das Programm kann sich Details über die Dauer der Ausführung der einzelnen Aufgaben speichern.
\subitem \textbf{Wichtigkeit} Optional
\subitem \textbf{Details} Die Informationen über die Arbeitszeit einer Aufgabe sind aufgeteilt in:\newline
1. aktive Zeit: Die Zeit, in der die Aufgabe tatsächlich ausgeführt wurde.\newline
2. passive Zeit: Die Zeit in der die Aufgabe in der Warteschlange verbracht hat.\newline
3. Gesamtzeit: Die Zeit die zwischen Auftragseingang und Abschicken der Abschlussbenachrichtigung vergangen ist.
\end{minipage}
\pagebreak

% Auf wie vielen Rechnern lief die Aufgabe (Wie oft wurde sie pausiert)
\begin{minipage}[t]{\linewidth}
\item[PD42] \textbf{Pausen}
\subitem \textbf{Erklärung} Anzahl der Pausen und möglicherweise Rechnerwechsel einer Aufgabe.
\subitem \textbf{Wichtigkeit} Optional (Statistik)
\subitem \textbf{Details} Für jedes Programm wird die Anzahl der erfolgten Pausen abgespeichert.\newline
Zusätzlich soll hinterlegt werden, auf welchen Rechnern die Aufgabe aktiv lief.
\end{minipage}
\pagebreak

% Globale Statistiken wie z.B. Anzahl aktive, passive, gequete Aufgaben, Gesamtlast, ...
\begin{minipage}[t]{\linewidth}
\item[PD60] \textbf{Globale Statistiken}
\subitem \textbf{Erklärung} Das Programm soll in der Lage sein, Kennwerte bezüglich des Gesamtsystems (alle Aufgaben) zu erheben.
\subitem \textbf{Wichtigkeit} Optional
\subitem \textbf{Details} Die einzelnen Kennwerte sind im folgenden Aufgelistet.\newline
Ein globaler Kennwert enthält eine Information über das gesamte System. Das heißt über den Zustand aller Aufgaben, aller Worker, usw.
\end{minipage}
\pagebreak

% Eine Liste mit allen verfügbaren Workern
\begin{minipage}[t]{\linewidth}
\item[PD61] \textbf{Worker Liste}
\subitem \textbf{Erklärung} Eine Liste aller einsatzbereiten Worker.
\subitem \textbf{Wichtigkeit} Nicht-optional
\subitem \textbf{Details} Enthält alle verfügbaren Worker.\newline
Somit kann der Server die Aufgaben auf alle Worker verteilen.
\end{minipage}
\pagebreak

%Anzahl aktive und passive Aufgaben
\begin{minipage}[t]{\linewidth}
\item[PD62] \textbf{Aufgaben}
\subitem \textbf{Erklärung} Anzahl aller laufenden und wartenden Aufgaben.
\subitem \textbf{Wichtigkeit} Optional
\subitem \textbf{Details} Die Anzahl der derzeit laufenden (aktiven) Aufgaben und die Anzahl der in der Warteschlange verweilenden Aufgaben.
\end{minipage}
\pagebreak

%Gesamtlast
\begin{minipage}[t]{\linewidth}
\item[PD63] \textbf{Gesamtlast}
\subitem \textbf{Erklärung} Auslastung der Worker
\subitem \textbf{Wichtigkeit} Optional
\subitem \textbf{Details} Gibt einen Wert für die Gesamtauslastung der Worker (CPU, Arbeitsspeicher, usw.)
\end{minipage}
\pagebreak

%Prioritätsverteilung
\begin{minipage}[t]{\linewidth}
\item[PD64] \textbf{\glspl{Prioritaet} der Aufgaben}
\subitem \textbf{Erklärung} Summe aller Aufgaben einer bestimmten \gls{Prioritaet}
\subitem \textbf{Wichtigkeit} Optional
\subitem \textbf{Details} Summiert für jede \gls{Prioritaet} die Anzahl der derzeit aktiven und passiven Aufgaben auf, die mit dieser \gls{Prioritaet} gestartet wurden.
\end{minipage}
\pagebreak

\end{itemize}

\subsection{Personendaten}
\subsection{Messdaten}

\section{Nichtfunktionale Anforderungen}
\begin{itemize}[nosep]
\leftskip=0.5cm

\begin{comment}

%Format einer nichtfunktionalen Anforderung:
\begin{minipage}[t]{\linewidth}
\item[FA00] \textbf{<Titel>}
\subitem \textbf{Erklärung} <In ca. 3 Zeilen eine grobe Beschreibung geben>
\subitem \textbf{Wichtigkeit} <Wie relevant ist es für uns, diese Anforderung zu erfüllen>
\subitem \textbf{Bezug} <Welcher funktionalen Anforderung ist diese Anforderung zuzuordnen>
\subitem \textbf{Details} <Ausführliche Beschreibung dieser nichtfunktionalen Anforderung>
\end{minipage}
\pagebreak
%Ende der Vorlage

\end{comment}

\item[NF10]
\end{itemize}

% Verteilung und Start der Aufgabe innerhalb von XX Minuten

% Verhältnis von Laufzeit und Pausierzeit soll nicht geringer sein als 1 zu XX

% Mindestens 1000 Aufgaben sollen in der Warteschlage gehalten werden können

% Mindestens 100 Nutzer sollen zeitgleich neue Aufgaben in Auftrag geben können

% Bei Ausfall der Worker (Stromausfall, ...) soll nicht mehr als 1 Stunde Rechenzeit verloren gehen -> Stündliche Backups

% Bei Abschluss einer Aufgabe soll die Rückmeldung innerhalb von XX Minuten erfolgen

% Statistiken sollen nur veröffentlicht werden, nachdem die Aufgabe abgeschlossen ist

% Ein Administrator soll den Prioritätenpool verwalten können

% Ein Benutzer darf nur auf eigene Dateien zugreifen

% Statistiken sind read only

\section{Systemmodelle}

% hier die Systemmodelle einfügen

\section{Benutzer Oberfläche}
\subsection{Befehlszeile}
\subsubsection{Server bzw. Arbeiter}
Startet das Programm im \gls{Daemon} Modus.
\begin{mycode}
bb -d [<--server|-s>]
\end{mycode}

\subsubsection{Aufgabe erstellen}
Sendet die angegebene Aufgabe an den Server und gibt die Aufgaben ID an der Standard Ausgabe aus (Dezimal).
\begin{mycode}
bb [--block|-b] [<--priority|-p> <Integer>] [<--docker-file|-df> <path to dockerfile>] [<--max-cpu-count|-Mc> <Integer>] [<--min-cpu-count|-mc> <Integer>] [<--max-ram|-Mr> <Integer>] [<--min-ram|-mr> <Integer>] <program> [args]+
\end{mycode}

\subsubsection{Aufgaben sichern}
Erstellt einen Schnapschuss der Aufgabe, der später bei bedarf wiederhergestellt werden kann.
Gibt die Sicherungs ID an der Standard Ausgabe aus (Dezimal).
\begin{mycode}
bb <--backup|-b> <id>
\end{mycode}
% \subsubsubsection{<id>} Zu viele ebenen
Aufgaben ID die beim erstellen der Aufgabe ausgegeben wurde.

\subsubsection{Aufgaben fortsetzen}
Setzt die angehaltene Aufgabe fort.
\begin{mycode}
bb <--continue|-c> <id>
\end{mycode}
\subsubsection{Aufgaben wiederherstellen}
\begin{mycode}
bb <--restore|-r> <id> <backupid>
\end{mycode}
% \subsubsubsection{<backupid>} Zu viele ebenen
Sicherungs ID die beim erstellen der Sicherung ausgegeben wurde.

\subsubsection{Aufgaben Pausieren}
Pausiert die Aufgabe mit der angegeben ID Nummer.
\begin{mycode}
bb <--pause|-p> <id>
\end{mycode}

\subsubsection{Aufgaben beenden}
Beendet die Aufgabe mit der angegeben ID Nummer.
\begin{mycode}
bb <--stop|-s> <id>
\end{mycode}

\subsubsection{\texttt{-{}-block}}
Kehrt erst nach der Ausführung der Aufgabe zum Aufrufer zurück. Nützlich um voneinander abhängige Aufgaben, in einem Konsolen Skript, nacheinander auszuführen.

\subsubsection{\texttt{-{}-priority}}
Legt die \gls{Prioritaet} der auszuführenden Aufgabe fest, gefolgt von low (4), medium (3), high (2), extreme (1), bananas (0).
Es kann der Name bzw. die Zahl in der Klammer als \gls{Prioritaet} verwendet werden.

\subsubsection{\texttt{-{}-docker-file}}
Legt sämtliche Eigenschaften eines Docker Abbildes, gefolgt vom Pfad einer benutzerdefinierten \href{https://docs.docker.com/engine/reference/builder/}{DockerFile}, fest.
Beispielsweise Benutzer ID, installierte Programme, auszuführende Umgebung.

\subsubsection{\texttt{-{}-min-cpu-count} und \texttt{-{}-max-cpu-count}}
Legen die minimale bzw. maximale anzahl der verwendbaren \glspl{CPU} fest.

\subsubsection{\texttt{-{}-min-ram} und \texttt{-{}-max-ram}}
Legen den minimal verfügbaren bzw. maximale verwendbaren Arbeitsspeicher fest.

\subsubsection{\texttt{-{}-server}}
Startet den Server, der die Aufgaben an die Arbeiter verteilt.

\section{Qualitätszielbestimmungen}
% Tabelle ... schafeln .. was ist ihm wichtig

\begin{comment}

%Format eines Testfalls:
\begin{minipage}[t]{\linewidth}
\item[FA00] \textbf{<Titel>}
\subitem \textbf{Erklärung} <Was soll getestet werden>
\subitem \textbf{Ablauf} <Vorbedingung/Startzustand>
<Eine Sequenz von Aktionen und Zwischenzuständen>
<Nachbedingung/Endzustand>
<Alle Zustände können auch wegfallen>
\end{minipage}
\pagebreak
%Ende der Vorlage

\end{comment}

\section{Testfälle/Testszenarien}
\subsection{Globale Testfälle}
\subsubsection{Grundlegende Testfälle}

\begin{itemize}
\item[T01] \textbf{Verbinden des Clients mit dem Server (FA1, FA2, FA3)}
\subitem \textbf{Erklärung} Ziel ist zu testen, ob sich der Client beim Programstart automatisch mit dem Server verbinden kann.
\subitem \textbf{Ablauf} Ausgegangen wird von einem bereits laufenden System, das mindestens aus dem Server und einem Arbeiter besteht.
Der Client führt folgenden Befehl aus:
\begin{mycode}
bb --block sleep 10 ; echo "test"
\end{mycode}
Bekommt der Client die Fehlermeldung
\begin{mycode}
Error: Can not find server
\end{mycode}
so konnte keine Verbindung zum Server hergestellt werden. Erhält er die Fehlermelung
\begin{mycode}
Error: Could not authenticate to the server
\end{mycode}
so konnte der Benutzer sich nicht gegenüber dem Server authentisieren. Erhält er eine andere Fehlermeldung, so konnte die Aufgabe nicht gestartet werden.
Bekommt er hingegen eine Job-ID zurück, so konnte die Aufgabe erfolgreich in die Warteschlange eingereiht werden.

\item[T02] \textbf{Speichern von Parametern in dem \gls{Configfile} (FA41 FA42)}
\subitem \textbf{Erklärung} Ziel ist zu testen, ob im \gls{Configfile} angegebene Einstellungen beim Erstellen von Aufgaben berücksichtigt werden.
\subitem \textbf{Ablauf} Ausgegangen wird von einem bereits funktionierenden System, in dem Aufgaben an Arbeiter verteilt werden können.
Der Nutzer trägt in das \gls{Configfile} eine maximale Speichernutzung von einem Kilobyte ein und erstellt eine Aufgabe, die 2 Kilobyte benötigt.
Bekommt er eine Benachrichtigung, dass das Programm abgestürzt ist, weil der Speicher vollgelaufen ist, wird das \gls{Configfile} korrekt mit einbezogen.

\item[T03] \textbf{Festlegen von Prioritäten (FA 43)}
\subitem \textbf{Erklärung} Ziel ist zu testen, ob Aufgaben mit einer höheren \gls{Prioritaet} bevorzugt werden.
\subitem \textbf{Ablauf} Ausgegangen wird von einem bereits funktionierenden System mit genau einem Arbeiter, in dem Aufgaben verteilt und bearbeitet werden können.
Der Nutzer startet eine Aufgabe die einige Zeit benötigt. Während der Bearbeitung gibt er eine Aufgabe mit einer niedrigen \gls{Prioritaet} auf gefolgt von derselben Aufgabe mit einer hohen Priorität.
Sind alle drei Aufgaben bearbeitet, startet er wieder eine Aufgabe, die einige Zeit benötigt. Danach erstellt er eine Aufgabe mit einer hohen \gls{Prioritaet} und danach dieselbe Aufgabe mit niedriger Priorität.
Wenn das System korrekt funktioniert, werden in beiden Fällen die Aufgaben mit hoher \gls{Prioritaet} zuerst bearbeitet.

\item[T04] \textbf{Festlegen von minimal und maximal bereitgestellten CPU-Kernen (FA44)}
\subitem \textbf{Erklärung} Ziel ist zu testen, ob der Nutzer festlegen kann wie viele Kerne einer Aufgabe zur Verfügung gestellt werden sollen.
\subitem \textbf{Ablauf} Da jede Aufgabe in einem eigenen Container ausgeführt wird, reicht es aus eine Aufgabe zu starten, die die Anzahl der CPU-Kerne der virtuellen Umgebung ausgibt und diese Aufgabe mit verschiedenen Werten aufgibt.

\item[T05] \textbf{Festlegen von maximal nutzbarem RAM (FA45)}
\subitem \textbf{Erklärung} Ziel ist zu testen, ob der Nutzer festlegen kann wie viel Hauptspeicher seiner Anwendung zur Verfügung gestellt werden soll.
\subitem \textbf{Ablauf} Ausgegangen wird von einem funktionierenden System, in dem der Nutzer Aufgaben erstellen kann, die bearbeitet werden.
Der Nutzer startet eine Aufgabe, die mindestens 2 Kilobyte Speicher anfordert und stellt der Aufgabe einen maximalen Hauptspeicher von einem Kilobyte zur Verfügung.
Bekommt er eine Nachricht, dass der Speicher vollgeöaufen ist und die Aufgabe deshalb abgebrochen werden musste, funktioniert das System.

\end{itemize}

\subsubsection{Erweiterte Testfälle}
\subsection{Testszenarien}

\clearpage
\printnoidxglossaries

\end{document}
