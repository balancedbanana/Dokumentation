\documentclass[a4paper,12pt]{article}
\usepackage{amssymb} % needed for math
\usepackage{amsmath} % needed for math
\usepackage[utf8]{inputenc} % this is needed for german umlauts
\usepackage[ngerman]{babel} % this is needed for german umlauts
\usepackage[T1]{fontenc}    % this is needed for correct output of umlauts in pdf
\usepackage[margin=2.5cm]{geometry} %layout
\usepackage{booktabs}
\usepackage{hyperref}
\hypersetup{pdftitle={Balanced Banana},bookmarks=true,}
\usepackage{graphicx}
\usepackage{csquotes}
\usepackage[nonumberlist]{glossaries}
\usepackage{enumitem}
\usepackage{verbatim}

\deftranslation[to=ngerman]{Glossary}{\section{Stichwortverzeichnis}}

\makeatletter
\newenvironment{mycode}
 {\def\@xobeysp{\ }\verbatim\rightskip=0pt plus 6em\relax}
 {\endverbatim}
\makeatother

\setitemize{align=parleft, labelsep=0.5cm}

\makenoidxglossaries

\newglossaryentry{CLI}
{
	name=CLI,
	description={Befehlszeile, engl. Commandline Interface},
}

\newglossaryentry{CPU}
{
	name=CPU,
	description={Central Processing Unit, Kern jedes Rechners um Anwendungen auszuführen},
}

\newglossaryentry{Daemon}
{
	name=Daemon,
	description={Ein Dienst der auf Anfragen reagiert und antwortet},
}
\newglossaryentry{Client}
{
	name={Client},
	description={Programm auf dem Computer eines Benutzers um mit dem Server zu  kommunizieren}
}

\newglossaryentry{Configfile}
{
	name={Configfile},
	description={Eine Datei die bestimmte Einstellungen speichert}
}

\newglossaryentry{Server}
{
	name={Server},
	description={Programm auf dem Server, der die Benutzer (Außenwelt) mit den Arbeitern (Privates Netzwerk) verbindet.}
}

\newglossaryentry{Benutzer}
{
	name={Benutzer},
	description={Eine Person, die dazu in der Lage ist, Befehle auszuführen.}
}

\newglossaryentry{Multicast}
{
	name={Multicast},
	description={Eine besondere Form von Netzwerk Packet, welches an alle Adressen des Netzwerkes gesendet wird.}
}

\newglossaryentry{Worker}
{
	name={Worker},
	description={Programm auf den Rechnern, welche dafür zuständig sind, die Aufgaben auszuführen.}
}

\newglossaryentry{Administrator}
{
	name={Administrator},
	description={Eine Person, die dazu privilegiert ist, das Gesamtsystem zu verwalten und über uneingeschränkten Zugriff auf Ein- und Ausgabedateien verfügt.}
}

\title{Balanced Banana}
\author{Niklas Lorenz \and Thomas Häuselmann \and Rakan Zeid Al Masri \and Christopher Lukas Homberger \and Jonas Seiler}


%%%%%%%%%%%%%%%%%%%%%%%%%%%%%%%%%%%%%%%%%%%%%%%%%%%%%%%%%%%%%%%%%%%%%%
% Create a shorter version for tables. DO NOT CHANGE               	 %
%%%%%%%%%%%%%%%%%%%%%%%%%%%%%%%%%%%%%%%%%%%%%%%%%%%%%%%%%%%%%%%%%%%%%%
\newcommand\addrow[2]{#1 &#2\\ }

\newcommand\addheading[2]{#1 &#2\\ \hline}
\newcommand\tabularhead{\begin{tabular}{lp{13cm}}
\hline
	}

\newcommand\addmulrow[2]{ \begin{minipage}[t][][t]{2.5cm}#1\end{minipage}%
   &\begin{minipage}[t][][t]{8cm}
    \begin{enumerate} #2   \end{enumerate}
    \end{minipage}\\ }

\newenvironment{usecase}{\tabularhead}
{\hline\end{tabular}}

\usepackage{microtype}

\begin{document}
\pagenumbering{roman}
\begin{titlepage}
    \begin{center}
    
     \vspace*{0.8cm}
 
        \includegraphics[width=0.5\textwidth]{balancedbanana}
        \vspace*{1cm}
 
        \Huge
        \textbf{Balanced Banana}
 
        \vspace{0.5cm}
        \LARGE
        A Distributed Task Scheduling System
        
        \vspace{0.5 cm}
        \LARGE
        Implementierung
 
        \vspace{1.5cm}

        \large
        \textbf{Niklas Lorenz, Thomas Häuselmann, Rakan Zeid Al Masri, Christopher Lukas Homberger und Jonas Seiler}
 
        \vspace*{0.5cm}

        \textbf{\today}
 
       
        
 
    \end{center}
\end{titlepage}         % Deckblatt.tex laden und einfügen
\setcounter{page}{2}
\tableofcontents          % Inhaltsverzeichnis ausgeben
\clearpage
\pagenumbering{arabic}

\section{Einleitung}
\vspace*{1.5cm}
Aufgabenverteilung ist ein in vielen Unternehmen übliches Problem. Wenn ein Unternehmen größer wird, 
so werden auch die verfügbaren Rechenressourcen und die darauf ausgeführten Aufgaben größer und komplexer sein. Es wird immer schwieriger, die genannten Ressourcen effizient und gerecht auf die verschiedenen Teams, Mitarbeiter und Arbeitsplätze aufzuteilen. \\

Viele Lösungen gibt es bereits auf dem Markt, sie sind aber zu komplex und nicht leicht erweiterbar. Balanced Banana löst diese Probleme in Form eines kompakten, einfach zu bedienenden und skalierbaren Programms. \\

Mit unsrer Programm kann der Benutzer seine Aufgabe einfach von der Kommandozeile absetzen. Darüber hinaus ist der Benutzer durch die Verwendung von Parametern in der Lage, seinem Programm zusätzliche Einschränkungen und/oder Bedingungen hinzuzufügen. Mit Hilfe intelligenter Algorithmen ist unser Programm in der Lage, die verfügbaren Rechenressourcen für Aufgaben effizient zu verteilen, basierend auf Größe, Priorität, Einschränkungen und Bedingungen.

\section{Zielbestimmung}
\subsection{Musskriterien}
\subsection{Wunschkriterien}
\subsection{Abgrenzungskriterien}
% Was will ich bewusst nicht umsetzen?
% Was soll es nicht sein?

\section{Produkteinsatz}
% Zielgruppe
% Anwendungsbereiche
% Betriebsbedinugen
% Wer? Was? Wozu?

\section{Produktumgebung}
% Unter welcher Software / Hardware läuft es?

\section{Funktionale Anforderungen}
\subsection{Liste der funktionalen Anforderungen}

\subsubsection{Kernanforderungen} %Name ändern pls

\begin{itemize}[nosep]
\leftskip=0.5cm

\begin{comment}

%Format einer funktionalen Anforderung:
\begin{minipage}[t]{\linewidth}
\item[FA00] \textbf{<Titel>}
\subitem \textbf{Erklärung} <In ca. 3 Zeilen eine grobe Beschreibung geben>
\subitem \textbf{Wichtigkeit} <entweder Kern-Funktionalität oder Optionale-Funktionalität>
\subitem \textbf{Voraussetzung(en)} <Wann ist diese Funktion nutzbar?> <dieser Punkt kann weggelassen werden>
\subitem \textbf{Nachbedingung(en)} <Dieser Punkt kann weggelassen werden>
\subsubitem \textbf{Erfolg} <Was geschieht wenn diese Funktion erfolgreich ausgeführt wurde>
\subsubitem \textbf{Misserfolg} <Was geschieht wenn diese Funktion nicht ausgeführt werden kann>
\subitem \textbf{Auslöser} <Wie wird diese Funktion gestartet> <Dieser Punkt kann weggelassen werden>
\subitem \textbf{Details} <Ausführliche Beschreibung dieser funktionalen Anforderung>
\end{minipage}
\pagebreak
%Ende der Vorlage

\end{comment}

%\item[FA10]	\gls{Client} verbindet sich beim Starten mit dem Server
\begin{minipage}[t]{\linewidth}
\item[FA10] \textbf{Automatisierter Verbindungsaufbau}
\subitem \textbf{Erklärung} Der \gls{Client} soll ohne explizite Aufforderung durch den \gls{Benutzer} mit dem \gls{Server} eine Netzwerk Verbindung zum Übermitteln der Daten aufbauen.
\subitem \textbf{Wichtigkeit} Kernfunktionalität
\subitem \textbf{Voraussetzung(en)} Ein \gls{Server} muss erreichbar sein.
\subitem \textbf{Nachbedingung(en)}
\subsubitem \textbf{Erfolg} Der \gls{Client} ist in der Lage jeden Befehlsaufruf, ohne den \gls{Benutzer} darüber zu informieren, an den \gls{Server} zu übermitteln.
\subsubitem \textbf{Misserfolg} Eine Fehlernachricht wird ausgegeben, die den \gls{Benutzer} darüber in Kenntnis setzt, dass keine Netzwerk Verbindung mit dem \gls{Server} aufgebaut werden konnte.
\subitem \textbf{Auslöser} Jeder Befehlsaufruf des \gls{Client} löst einen automatisierten Verbindungsaufbau aus.
\subitem \textbf{Details} Zur einfachen Verwendung eines \gls{Client} soll sich dieser selbstständig mit einem \gls{Server} verbinden.\newline
Die Verbindung wird zum Start jeder Übermittlung aufgebaut und nach Ende der Übermittlung geschlossen.\newline
Die Verbindung dient einzig dem Zweck, den \gls{Server} über einen Befehlsaufruf  in Kenntnis zu setzen.\newline
Ablauf:\newline
Schritt 1: Der \gls{Client} sendet eine \gls{Multicast} Nachricht an das Netzwerk.\newline
Schritt 2: Der \gls{Server} beantwortet die Nachricht und setzt so den \gls{Client} über seine Existenz und Netzwerk-Adresse in Kenntnis.\newline
Schritt 3: Der \gls{Client} übermittelt den Befehlsaufruf an den Server.\newline
Schritt 4: Der \gls{Server} sendet eine Eingangsbestätigung an den \gls{Client}.\newline
Schritt 5: Da nun keine Nachrichten mehr übermittelt werden müssen, wird die Verbindung geschlossen.
\end{minipage}
\pagebreak

\begin{minipage}[t]{\linewidth}
\item[FA11] \textbf{Gedächtnis der vorherigen Verbindung}
\subitem \textbf{Erklärung} Der \gls{Client} soll sich Informationen über eine erfolgreiche Verbindung zu einem \gls{Server} merken.
\subitem \textbf{Wichtigkeit} Optional
\subitem \textbf{Voraussetzung(en)} Diese Funktionalität wird stets in Verbindung mit Funktion FA10 verwendet.\newline
Funktion FA10 konnte erfolgreich abgeschlossen werden.
\subitem \textbf{Nachbedingung(en)}
\subsubitem \textbf{Erfolg} Der \gls{Client} ist bei dem nächsten Verbindungsaufbau in der Lage, den \gls{Server} direkt anzusprechen.
\subsubitem \textbf{Misserfolg} Der \gls{Client} hat keine Informationen über einen existierenden \gls{Server}.
\subitem \textbf{Auslöser} Funktion FA10 wurde erfolgreich ausgeführt.
\subitem \textbf{Details} Der Verbindungsaufbau zwischen \gls{Client} und \gls{Server} mithilfe einer \gls{Multicast} Nachricht sollte vermieden werden. Daher soll sich der \gls{Client} den letzten \gls{Server} mit dem er sich erfolgreich verbunden hat merken. Somit kann der \gls{Client} bei dem nächsten Verbindungsaufbau die \gls{Multicast} Nachricht vermeiden, indem er sich direkt mit dem \gls{Server} in Verbindung setzt.\newline
Ist der hinterlegte \gls{Server} nicht mehr erreichbar, so wird er vom \gls{Client} vergessen.\newline
Der Ablauf von Funktionalität FA10 ändert sich wie folgt:\newline
Schritt 1: Der \gls{Client} versucht den vorgemerkten \gls{Server} zu erreichen. Ist dies nicht möglich, vergisst er diesen und sendet eine \gls{Multicast} Nachricht an das Netzwerk.\newline
Schritt 2-5: Unverändert
Schritt 6: Der \gls{Client} speichert sich die Netzwerkadresse des \gls{Server} ab.
\end{minipage}
\pagebreak

%\item[FA20] Benutzer kann eine Aufgabe über \gls{Client} einreihen
\begin{minipage}[t]{\linewidth}
\item[FA20] \textbf{Warteschlange}
\subitem \textbf{Erklärung} Sollte zum Zeitpunkt des Eingangs einer neuen Aufgabe kein \gls{Worker} mit hinreichend Arbeitskapazität verfügbar sein, so soll die Aufgabe nicht abgelehnt oder verloren gehen, sondern in eine Warteschlange eingereiht werden. Zu einem späteren Zeitpunkt kann die Aufgabe dann ausgeführt werden.
\subitem \textbf{Wichtigkeit} Kernfunktionalität
\subitem \textbf{Voraussetzung(en)} Diese Funktion ist auf jedem \gls{Server} verfügbar.
\subitem \textbf{Nachbedingung(en)}
\subsubitem \textbf{Erfolg} Die Aufgabe ist in einer Warteschlange eingereiht.
\subsubitem \textbf{Misserfolg} Der Auftraggeber (\gls{Client}) wird darüber in Kenntnis gesetzt, dass seine Aufgabe nicht bearbeitet wird.\newline
Ein \gls{Administrator} wird davon in Kenntnis gesetzt, dass ein Fehler beim Aufnehmen einer Aufgabe in die Warteschlange aufgetreten ist.
\subitem \textbf{Auslöser} Der \gls{Server} erhält eine Aufgabe zu einem Zeitpunkt, an dem kein geeigneter \gls{Worker} verfügbar ist.
\subitem \textbf{Details} Alle von einem \gls{Benutzer} in Auftrag gegebenen Aufgaben sollen angenommen und ausgeführt werden. Dies ist jedoch aufgrund beschränkter Rechenkapazität nicht immer sofort möglich. Daher sollen Aufgaben, die nicht sofort bearbeitet werden können, vorübergehend vom \gls{Server} in einer Warteschlange gehalten werden. Nachdem ausreichend Rechenkapazität frei geworden ist, soll die Warteschlange wieder geleert werden.
\end{minipage}
\pagebreak

%\item[FA30] Benutzer kann Parameter übergeben %evtl auf FA20 verweisen
\begin{minipage}[t]{\linewidth}
\item[FA30] \textbf{Befehlsparameter}
\subitem \textbf{Erklärung} Dem Programm können per Befehlszeile bestimmte Optionen bzw. Informationen bezüglich der Ausführung des gegebenen Befehls bereitgestellt werde.
\subitem \textbf{Wichtigkeit} Kernfunktionalität
\subitem \textbf{Voraussetzung(en)} Jeder Befehl kann mit einem gewissen Satz von Parametern aufgerufen werden.\newline
Die Parameter sind für den angegebenen Befehl zulässig.\newline
Es sind keine Parameter angegeben, die sich gegenseitig ausschließen.\newline
Jeder Parameter wird mit einem zulässigen Eingabewert angegeben.
\subitem \textbf{Nachbedingung(en)}
\subsubitem \textbf{Erfolg} Der Befehl wird unter den angegebenen Parametern ausgeführt.
\subsubitem \textbf{Misserfolg} Dem Benutzer wird mitgeteilt, welcher Parameter unzulässig angegeben wurde.\newline
Der Befehl wird nicht ausgeführt und nicht an den \gls{Server} weitergeleitet bzw. von diesem ignoriert.
\subitem \textbf{Auslöser} Angabe des Parameters auf der Kommandozeile.
\subitem \textbf{Details} Damit das Programm mehr als eine Funktion erfüllen kann oder eine der angebotenen Funktionen mit nicht-standardmäßigen Werten ausführen kann, muss dem Programm mitgeteilt werden, welche Funktion mit welchen Werten gewünscht ist. Hierzu werden Befehlsparameter verwendet (für genauere Angaben zu allen Parametern siehe nachfolgende funktionale Anforderungen sowie Absatz 10 Benutzeroberfläche).\newline
Jeder Parameter kann auf der Kommandozeile durch die Syntax "--<Name>" oder "-<Kürzel>", gefolgt von dem gewünschten Wert, verwendet werden.\newline
Ein Parameter hat entweder durch Angabe eines Wertes oder durch Angabe des Parameters Einfluss auf die Ausführung des Befehls (abhängig vom konkreten Parameter).\newline
Jeder Parameter hat einen eindeutig definierten Standardwert.\newline
Es ist möglich, dass zwei Parameter nicht in dem selben Befehl verwendet werden können und dürfen. Sollte eine solche Kollision dennoch vorhanden sein, schlägt die Ausführung des Befehls fehl.
\end{minipage}
\pagebreak

%\item[FA31]	\gls{Client} speichert voreingestellte Parameter in \gls{Configfile}
\begin{minipage}[t]{\linewidth}
\item[FA31] \textbf{Global definierte Standardwerte}
\subitem \textbf{Erklärung} Jeder Parameter hat einen global definierten Standardwert. Dieser wird verwendet, wenn der \gls{Benutzer} keinen anderen Wert angibt. Ein globaler Standardwert kann nie einen Fehler auslösen.
\subitem \textbf{Wichtigkeit} Kernfunktionalität
\subitem \textbf{Voraussetzung(en)} Dem Parameter kann ein Wert zugewiesen werden.
\subitem \textbf{Nachbedingung(en)} 
\subsubitem \textbf{Erfolg} Der Parameter wird mit einem unkritischen Wert aufgefüllt.\newline
Der Befehl wird trotz unvollständiger Parameter ausgeführt.
\subitem \textbf{Auslöser} Der Benutzer hat keinen eigenen Parameterwert angegeben.
\subitem \textbf{Details} Parameter, bei denen die reine Angabe auf der Kommandozeile nicht Aussagekräftig ist (Der Parameter erfordert z.B. einen numerischen Wert) müssen immer mit einem Wert ausgefüllt werden.\newline
Oftmals ist es lästig einen Parameter explizit anzugeben, da bis auf Randfälle immer der gleiche Wert verwendet wird. In solch einem Fall soll es möglich sein, den Parameter implizit anzugeben.\newline
Sollte ein \gls{Benutzer} vergessen, einen Parameter anzugeben, so soll der Befehl dennoch mit einem stets unkritischen Wert ausgeführt werden.\newline
Sollte ein \gls{Benutzer} nicht darüber im klaren sein, wozu ein bestimmter Parameter dient, soll dem \gls{Benutzer} ein unkritischer Wert vorgeschlagen werden.\newline
Die Standardwerte sind in einer Konfigurationsdatei abgespeichert.
\end{minipage}
\pagebreak

%\item[FA32]	Benutzer kann Parameter über \gls{Configfile} übergeben
\begin{minipage}[t]{\linewidth}
\item[FA32] \textbf{Benutzerdefinierte Standardwerte}
\subitem \textbf{Erklärung} Jeder Parameter hat einen eindeutig definierten Standardwert, der bei Nichtangabe des Parameters auf der Kommandozeile verwendet wird. Der \gls{Benutzer} kann lokal eine eigene Standardbelegung definieren.
\subitem \textbf{Wichtigkeit} Optional
\subitem \textbf{Voraussetzung(en)} Die lokal definierte Standardbelegung darf keine unzulässige Belegung enthalten.\newline
Der Benutzer hat keinen eigenen Parameterwert angegeben.
\subitem \textbf{Nachbedingung(en)}
\subsubitem \textbf{Erfolg} Bei Nichtangabe eines Parameters auf der Kommandozeile werden die von dem \gls{Benutzer} angegebenen Standardwerte vor den globalen Standardwerten eingesetzt.
\subitem \textbf{Auslöser} Der \gls{Benutzer} hat den Parameter nicht auf der Kommandozeile angegeben.
\subitem \textbf{Details} Ein \gls{Benutzer} verwendet möglicherweise bei jedem Befehlsaufruf immer identische Werte für manche Parameter. Sollten diese Werte nicht mit den global definierten Standardwerten übereinstimmen, kann der \gls{Benutzer} individuelle Standardwerte angeben. Somit muss der \gls{Benutzer} nicht bei jedem Aufruf den Parameter explizit angeben.\newline
Die benutzerdefinierten Standardwerte sind in einer ausgezeichneten Konfigurationsdatei abgespeichert.
\end{minipage}
\pagebreak

\item[FA33] Benutzer kann Priorität einer Aufgabe festlegen %Glossar Priorität???

\item[FA34] Benutzer kann minimale und maximale Anzahl genutzer Kerne festlegen %Glossar Kerne

\item[FA35] Benutzer kann maximal nutzbaren RAM festlegen %Glossar RAM

\item[FA40] Anfragen nach Status

\item[FA50] Benutzer bekommt Benachrichtigung über erledigte Aufgabe 

\end{itemize}

\subsubsection{optionale Anforderungen}
\begin{itemize}[nosep]
\leftskip=0.5cm
\item[OFA01] Benutzer kann eine geschätzte Restzeit einer Aufgabe sehen	
\end{itemize}
% Was soll das Produkt machen können

% Es soll einen Webserver als Mittelmann geben

% Es soll möglich sein, Backups von laufenden Aufgaben zu erstellen

% Aufgaben, die nach 2 Tagen arbeitszeit noch nicht beenden, sollen unterbrochen werden und dem Auftraggeber soll eine Mitteilung geschickt werden

% Die Anwendung soll über verschiedene Betriebsmodi verfügen (Client, Server, Worker, Admin?)

% Die folgenden Parameter sind für den Befehl vorgesehen:

% Priorität (low, medium, high, extreme, bananas)

% Anzahl der zu gewünschten Prozessorkerne

% Größe des benötigten Arbeitsspeichers

% Auf welchem Betriebssystem soll die Aufgabe ausgeführt werden?

% Ist die Aufgabe pausierbar?

% Soll der Aufrufer im Terminal blockiert werden

% An wen soll die Rückmeldung erfolgen (EMail oder Nutzerkonto)

% Wo sind die Ordner mit den benötigten Daten abgelegt (Pfad übersetzten damit der Worker die Ordner findet)

% Anforderung eines manuellen Backups / Pausierung / Abbruch

% Anforderung der Log daten

% Am Ende des unseres Befehls folgt der Befehl mit dem die Aufgabe gestartet werden kann

% Wie kann die Aufgabe abgebrochen oder pausiert werden (Einen Befehl spezifizieren)

\section{Produktdaten}
\begin{itemize}[nosep]
\leftskip=0.5cm
\item[PD10]
\end{itemize}
% Was soll gespeichert werden?

% Daten über ein Nutzerkonto?

% Ausführungszeit (insgesamt, aktiv, passiv)

% Ausführungsbefehl, Pausierbefehl, Abbruchbefehl

% Ist das Programm pausierbar

% Ausgabedaten

% Logdaten

% Mit welcher Priorität ist die Aufgabe gestartet worden

% wer hat die Aufgabe gestartet (EMail oder Nutzerkonto)

% Auf welchem Betriebssystem soll die Aufgabe ausgeführt werden

% Auf wie vielen Rechnern lief die Aufgabe (Wie oft wurde sie pausiert)

% Allgemein alle Parameter unseres Befehls die für diese Aufgabe angegeben wurden

% Globale Statistiken wie z.B. Anzahl aktive, passive, gequete Aufgaben, Gesamtlast, ...

% Eine Liste mit allen verfügbaren Workern

\subsection{Personendaten}
\subsection{Messdaten}

\section{Nichtfunktionale Anforderungen}
\begin{itemize}[nosep]
\leftskip=0.5cm
\item[NF10]
\end{itemize}

% Verteilung und Start der Aufgabe innerhalb von XX Minuten

% Verhältnis von Laufzeit und Pausierzeit soll nicht geringer sein als 1 zu XX

% Mindestens 1000 Aufgaben sollen in der Warteschlage gehalten werden können

% Mindestens 100 Nutzer sollen zeitgleich neue Aufgaben in Auftrag geben können

% Bei Ausfall der Worker (Stromausfall, ...) soll nicht mehr als 1 Stunde Rechenzeit verloren gehen -> Stündliche Backups

% Bei Abschluss einer Aufgabe soll die Rückmeldung innerhalb von XX Minuten erfolgen

% Statistiken sollen nur veröffentlicht werden, nachdem die Aufgabe abgeschlossen ist

% Ein Administrator soll den Prioritätenpool verwalten können

% Ein Benutzer darf nur auf eigene Dateien zugreifen

% Statistiken sind read only

\section{Systemmodelle}

\subsection{Szenarien}

\subsection{Anwendungsfälle}

\section{Benutzer Oberfläche}
\subsection{Befehlszeile}
\subsubsection{Client}
\begin{mycode}
bb [--block|-b] [<--priority|-p> <Integer>] [<--docker-file|-df> <path to dockerfile>] [<--max-cpu-count|-Mc> <Integer>] [<--min-cpu-count|-mc> <Integer>] [<--max-ram|-Mr> <Integer>] [<--min-ram|-mr> <Integer>] <program> [args]+
\end{mycode}

\subsubsection{Server bzw. Arbeiter}
Startet das Programm im \gls{Daemon} Modus.
\begin{mycode}
bb -d [<--server|-s>]
\end{mycode}

\subsubsection{\texttt{-{}-block}}
Kehrt erst nach der Ausführung der Aufgabe zum Aufrufer zurück. Nützlich um voneinander abhängige Aufgaben, in einem Konsolen Skript, nacheinander auszuführen.

\subsubsection{\texttt{-{}-priority}}
Legt die Priorität der auszuführenden Aufgabe fest, gefolgt von low (0), medium (1), high (2), banana (3).
Es kann der Name bzw. die Zahl in der Klammer als Priorität verwendet werden.

\subsubsection{\texttt{-{}-docker-file}}
Legt sämtliche Eigenschaften eines Docker Abbildes, gefolgt vom Pfad einer benutzerdefinierten \href{https://docs.docker.com/engine/reference/builder/}{DockerFile}, fest.
Beispielsweise Benutzer ID, installierte Programme, auszuführende Umgebung.

\subsubsection{\texttt{-{}-min-cpu-count} und \texttt{-{}-max-cpu-count}}
Legen die minimale bzw. maximale anzahl der verwendbaren \glspl{CPU} fest.

\subsubsection{\texttt{-{}-min-ram} und \texttt{-{}-max-ram}}
Legen den minimal verfügbaren bzw. maximale verwendbaren Arbeitsspeicher fest.

\subsubsection{\texttt{-{}-server}}
Startet den Server, der die Aufgaben an die Arbeiter verteilt.

\section{Qualitätszielbestimmungen}
% Tabelle ... schafeln .. was ist ihm wichtig

\section{Globale Testfälle und Szenarien}
\subsection{Globale Testfälle}
\subsubsection{Grundlegende Testfälle}
\subsubsection{Erweiterte Testfälle}
\subsection{Szenarien}
\section{Entwicklungsumgebung}

\clearpage
\printnoidxglossaries

\end{document}
