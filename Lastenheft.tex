\documentclass[a4paper,12pt]{article}
\usepackage{amssymb} % needed for math
\usepackage{amsmath} % needed for math
\usepackage[utf8]{inputenc} % this is needed for german umlauts
\usepackage[ngerman]{babel} % this is needed for german umlauts
\usepackage[T1]{fontenc}    % this is needed for correct output of umlauts in pdf
\usepackage[margin=2.5cm]{geometry} %layout
\usepackage{booktabs}
\usepackage{hyperref}
\hypersetup{pdftitle={Balanced Banana},bookmarks=true,}
\usepackage{graphicx}
\usepackage{csquotes}
\usepackage[nonumberlist]{glossaries}
\usepackage{enumitem}
\usepackage{verbatim}
\usepackage{indentfirst} % Adds indent for the first paragraph after a {/section}


\deftranslation[to=ngerman]{Glossary}{\section{Stichwortverzeichnis}}

\makeatletter
\newenvironment{mycode}
 {\def\@xobeysp{\ }\verbatim\rightskip=0pt plus 6em\relax}
 {\endverbatim}
\makeatother

\setitemize{align=parleft, labelsep=0.5cm}

\makenoidxglossaries

\newglossaryentry{CLI}
{
	name=CLI,
	description={Befehlszeile, engl. Commandline Interface},
}

\newglossaryentry{CPU}
{
	name=CPU,
	description={Central Processing Unit, Kern jedes Rechners um Anwendungen auszuführen},
}

\newglossaryentry{Daemon}
{
	name=Daemon,
	description={Ein Dienst der auf Anfragen reagiert und antwortet},
}
\newglossaryentry{Client}
{
	name={Client},
	description={Programm auf dem Computer eines Benutzers um mit dem Server zu  kommunizieren}
}
\newglossaryentry{Configfile}
{
	name={Configfile},
	description={Eine Datei die bestimmte Einstellungen speichert}
}

\title{Balanced Banana}
\author{Niklas Lorenz \and Thomas Häuselmann \and Rakan Zeid Al Masri \and Christopher Lukas Homberger \and Jonas Seiler}


%%%%%%%%%%%%%%%%%%%%%%%%%%%%%%%%%%%%%%%%%%%%%%%%%%%%%%%%%%%%%%%%%%%%%%
% Create a shorter version for tables. DO NOT CHANGE               	 %
%%%%%%%%%%%%%%%%%%%%%%%%%%%%%%%%%%%%%%%%%%%%%%%%%%%%%%%%%%%%%%%%%%%%%%
\newcommand\addrow[2]{#1 &#2\\ }

\newcommand\addheading[2]{#1 &#2\\ \hline}
\newcommand\tabularhead{\begin{tabular}{lp{13cm}}
\hline
	}

\newcommand\addmulrow[2]{ \begin{minipage}[t][][t]{2.5cm}#1\end{minipage}%
   &\begin{minipage}[t][][t]{8cm}
    \begin{enumerate} #2   \end{enumerate}
    \end{minipage}\\ }

\newenvironment{usecase}{\tabularhead}
{\hline\end{tabular}}

\usepackage{microtype}

\begin{document}
\pagenumbering{roman}
\begin{titlepage}
    \begin{center}
    
     \vspace*{0.8cm}
 
        \includegraphics[width=0.5\textwidth]{balancedbanana}
        \vspace*{1cm}
 
        \Huge
        \textbf{Balanced Banana}
 
        \vspace{0.5cm}
        \LARGE
        A Distributed Task Scheduling System
        
        \vspace{0.5 cm}
        \LARGE
        Implementierung
 
        \vspace{1.5cm}

        \large
        \textbf{Niklas Lorenz, Thomas Häuselmann, Rakan Zeid Al Masri, Christopher Lukas Homberger und Jonas Seiler}
 
        \vspace*{0.5cm}

        \textbf{\today}
 
       
        
 
    \end{center}
\end{titlepage}         % Deckblatt.tex laden und einfügen
\setcounter{page}{2}
\tableofcontents          % Inhaltsverzeichnis ausgeben
\clearpage
\pagenumbering{arabic}

\section{Einleitung}
\vspace*{1cm}

Aufgabenverteilung ist ein in vielen Unternehmen übliches Problem. Wenn ein Unternehmen größer wird, 
so werden auch die verfügbaren Rechenressourcen und die darauf ausgeführten Aufgaben größer und komplexer sein. Es wird immer schwieriger, die genannten Ressourcen effizient und gerecht auf die verschiedenen Teams, Mitarbeiter und Aufgaben aufzuteilen. \\

Viele Lösungen gibt es bereits auf dem Markt, sie sind aber zu komplex und nicht leicht erweiterbar. Balanced Banana löst diese Probleme in Form eines kompakten, einfach zu bedienenden und skalierbaren Programms. \\

Mit unserem Programm kann der Benutzer seine Aufgabe einfach von der Kommandozeile absetzen. Darüber hinaus ist der Benutzer durch die Verwendung von Parametern in der Lage, seinem Programm zusätzliche Einschränkungen und/oder Bedingungen hinzuzufügen. Mit Hilfe intelligenter Algorithmen ist unser Programm in der Lage, die verfügbaren Rechenressourcen für Aufgaben effizient zu verteilen, basierend auf Größe, Priorität, Einschränkungen und Bedingungen.

\section{Zielbestimmung}
\subsection{Musskriterien}
\subsection{Wunschkriterien}
\subsection{Abgrenzungskriterien}
% Was will ich bewusst nicht umsetzen?
% Was soll es nicht sein?

% Ist das nicht ne Wiederholung von Einleitung + Anforderungen?
% Ich würde das sehr kurz halten, so wie die Einleitung zum Beispiel

\section{Produkteinsatz}
% Zielgruppe
% Anwendungsbereiche
% Betriebsbedinugen
% Wer? Was? Wozu?

\section{Produktumgebung}
% Unter welcher Software / Hardware läuft es?

\section{Funktionale Anforderungen}

\subsection{Übersicht der Anforderungen}

\subsubsection{Kernanforderungen} %Name ändern pls

\begin{itemize}[nosep]
\leftskip=0.5cm

\begin{comment}

%Format einer funktionalen Anforderung:
\begin{minipage}[t]{\linewidth}
\item[FA00] \textbf{<Titel>}
\subitem \textbf{Erklärung} <In ca. 3 Zeilen eine grobe Beschreibung geben>
\subitem \textbf{Wichtigkeit} <entweder Kern-Funktionalität oder Optionale-Funktionalität>
\subitem \textbf{Vorraussetzungen} <Wann ist diese Funktion nutzbar?> <dieser Punkt kann weggelassen werden>
\subitem \textbf{Nachbedingung} <Dieser Punkt kann weggelassen werden>
\subsubitem \textbf{Erfolg} <Was geschieht wenn diese Funktion erfolgreich ausgeführt wurde>
\subsubitem \textbf{Misserfolg} <Was geschieht wenn diese Funktion nicht ausgeführt werden kann>
\subitem \textbf{Auslöser} <Wie wird diese Funktion gestartet> <Dieser Punkt kann weggelassen werden>
\subitem \textbf{Details} <Ausführliche Beschreibung dieser funktionalen Anforderung>
\end{minipage}
\pagebreak
%Ende der Vorlage

\end{comment}

% Soll nur eine Liste sein, dass Ausschreiben kommt später

\subsubsection{Kernanforderungen}

\begin{itemize}[nosep]
\leftskip=0.5cm

\item[FA1]	\gls{Client} verbindet sich beim Starten mit dem Server
\item[FA2] Benutzer authentifiziert sich über den \gls{Client} gegenüber dem Server
\item[FA3] Benutzer kann eine Aufgabe über \gls{Client} einreihen
\item[FA4] Benutzer kann Parameter übergeben %evtl auf FA20 verweisen
\item[FA41]	\gls{Client} speichert voreingestellte Parameter in \gls{Configfile}
\item[FA42]	Benutzer kann Parameter über \gls{Configfile} übergeben
\item[FA43] Benutzer kann Priorität einer Aufgabe festlegen %Glossar Priorität???
\item[FA44] Benutzer kann minimale und maximale Anzahl genutzer Kerne festlegen %Glossar Kerne
\item[FA45] Benutzer kann maximal nutzbaren RAM festlegen %Glossar RAM
\item[FA46] Benutzer kann das benutzte Betriebssystem festlegen
\item[FA47] Benutzer kann angeben ob die Aufgabe pausierbar ist. %evtl optional
\item[FA48] Benutzer kann angeben ob der \gls{Client} blockieren soll bis die Aufgabe beendet ist. %Blockierbar glosar?
\item[FA49] Benutzer übergibt Pfad zu den für die Aufgabe benötigten Dateien. %Pfad glosar?
\item[FA5] Benutzer kann den Status einer Aufgabe einsehen
\item[FA51] Benutzer kann die verstrichene Zeit einer Aufgabe einsehen %evtl optional
\item[FA6] Benutzer bekommt Benachrichtigung über erledigte Aufgabe
\item[FA7] Server erstellt regelmäßige Sicherungen von laufenden Aufgaben
\item[FA8] Benutzer kann die Ausgabe seiner Aufgabe anfordern
\end{itemize}

\subsubsection{optionale Anforderungen}
\begin{itemize}[nosep]
\leftskip=0.5cm
\item[OFA1] Benutzer kann eine geschätzte Restzeit einer Aufgabe sehen	
\item[OFA2] Server stoppt Aufgaben die zu lange dauern %Dauer einfügen
\item[OFA3] Benutzer kann eine manuelle Stoppung seiner Aufgabe anfordern
\item[OFA4] Benutzer kann eine manuelle Sicherung seiner Aufgabe anfordern
\end{itemize}
% Die Anwendung soll über verschiedene Betriebsmodi verfügen (Client, Server, Worker, Admin?)

% An wen soll die Rückmeldung erfolgen (EMail oder Nutzerkonto)

% Am Ende des unseres Befehls folgt der Befehl mit dem die Aufgabe gestartet werden kann
\subsubsection{Funktionale Anforderungen}

\begin{comment}

%Format einer funktionalen Anforderung:
\begin{minipage}[t]{\linewidth}
\item[FA00] \textbf{<Titel>}
\subitem \textbf{Erklärung} <In ca. 3 Zeilen eine grobe Beschreibung geben>
\subitem \textbf{Wichtigkeit} <entweder Kern-Funktionalität oder Optionale-Funktionalität> %würde ich weglassen
\subitem \textbf{Vorraussetzungen} <Wann ist diese Funktion nutzbar?> <dieser Punkt kann weggelassen werden> %absolut nicht, das ist wichtig für testfälle
\subitem \textbf{Nachbedingung} <Dieser Punkt kann weggelassen werden>
\subsubitem \textbf{Erfolg} <Was geschieht wenn diese Funktion erfolgreich ausgeführt wurde>
\subsubitem \textbf{Misserfolg} <Was geschieht wenn diese Funktion nicht ausgeführt werden kann>
\subitem \textbf{Auslöser} <Wie wird diese Funktion gestartet> <Dieser Punkt kann weggelassen werden>
\subitem \textbf{Details} <Ausführliche Beschreibung dieser funktionalen Anforderung>
\end{minipage}
\pagebreak
%Ende der Vorlage

\end{comment}

\begin{itemize}[nosep]
\leftskip=0.5cm
\item[FA1] \textbf{\gls{Client} verbindet sich beim Start mit dem Server.}
\begin{itemize}[nosep]
\item \textbf{Erklärung:} Beim Starten der Client-Anwendung versucht diese sich automatisch mit der Serveranwendung zu verbinden.
\item \textbf{Vorraussetzungen:} Keine.
\item \textbf{Erfolg:} Die Client-Anwendung konnte sich mit der Serveranwendung verbinden. Eine entsprechende Meldung wird ausgegeben.
\item \textbf{Misserfolg:} Die Client-Anwendung konnte sich nicht mit der Serveranwendung verbinden. Eine entsprechende Fehlermeldung wird ausgegeben.
\item \textbf{Auslöser:} Die Client-Anwendung wird gestartet.
\item \textbf{Details:}
\begin{itemize}[nosep]
	\item Wenn der Benutzer die Client-Anwendung startet, versucht diese die zugehörige Serveranwendung zu finden und sich mit dieser zu verbinden.
\end{itemize}
\end{itemize}
\end{itemize}


\section{Produktdaten}
\begin{itemize}[nosep]
\leftskip=0.5cm
\item[PD10]
\end{itemize}
% Was soll gespeichert werden?

% Daten über ein Nutzerkonto?

% Ausführungszeit (insgesamt, aktiv, passiv)

% Ausführungsbefehl, Pausierbefehl, Abbruchbefehl

% Ist das Programm pausierbar

% Ausgabedaten

% Logdaten

% Mit welcher Priorität ist die Aufgabe gestartet worden

% wer hat die Aufgabe gestartet (EMail oder Nutzerkonto)

% Auf welchem Betriebssystem soll die Aufgabe ausgeführt werden

% Auf wie vielen Rechnern lief die Aufgabe (Wie oft wurde sie pausiert)

% Allgemein alle Parameter unseres Befehls die für diese Aufgabe angegeben wurden

% Globale Statistiken wie z.B. Anzahl aktive, passive, gequete Aufgaben, Gesamtlast, ...

% Eine Liste mit allen verfügbaren Workern

\subsection{Personendaten}
\subsection{Messdaten}

\section{Nichtfunktionale Anforderungen}
\begin{itemize}[nosep]
\leftskip=0.5cm
\item[NF10]
\end{itemize}

% Verteilung und Start der Aufgabe innerhalb von XX Minuten

% Verhältnis von Laufzeit und Pausierzeit soll nicht geringer sein als 1 zu XX

% Mindestens 1000 Aufgaben sollen in der Warteschlage gehalten werden können

% Mindestens 100 Nutzer sollen zeitgleich neue Aufgaben in Auftrag geben können

% Bei Ausfall der Worker (Stromausfall, ...) soll nicht mehr als 1 Stunde Rechenzeit verloren gehen -> Stündliche Backups

% Bei Abschluss einer Aufgabe soll die Rückmeldung innerhalb von XX Minuten erfolgen

% Statistiken sollen nur veröffentlicht werden, nachdem die Aufgabe abgeschlossen ist

% Ein Administrator soll den Prioritätenpool verwalten können

% Ein Benutzer darf nur auf eigene Dateien zugreifen

% Statistiken sind read only

\section{Systemmodelle}

\subsection{Szenarien}

\subsection{Anwendungsfälle}

\section{Benutzer Oberfläche}
\subsection{Befehlszeile}
\subsubsection{Client}
\begin{mycode}
bb [--block|-b] [<--priority|-p> <Integer>] [<--docker-file|-df> <path to dockerfile>] [<--max-cpu-count|-Mc> <Integer>] [<--min-cpu-count|-mc> <Integer>] [<--max-ram|-Mr> <Integer>] [<--min-ram|-mr> <Integer>] <program> [args]+
\end{mycode}

\subsubsection{Server bzw. Arbeiter}
Startet das Programm im \gls{Daemon} Modus.
\begin{mycode}
bb -d [<--server|-s>]
\end{mycode}

\subsubsection{\texttt{-{}-block}}
Kehrt erst nach der Ausführung der Aufgabe zum Aufrufer zurück. Nützlich um voneinander abhängige Aufgaben, in einem Konsolen Skript, nacheinander auszuführen.

\subsubsection{\texttt{-{}-priority}}
Legt die Priorität der auszuführenden Aufgabe fest, gefolgt von low (0), medium (1), high (2), banana (3).
Es kann der Name bzw. die Zahl in der Klammer als Priorität verwendet werden.

\subsubsection{\texttt{-{}-docker-file}}
Legt sämtliche Eigenschaften eines Docker Abbildes, gefolgt vom Pfad einer benutzerdefinierten \href{https://docs.docker.com/engine/reference/builder/}{DockerFile}, fest.
Beispielsweise Benutzer ID, installierte Programme, auszuführende Umgebung.

\subsubsection{\texttt{-{}-min-cpu-count} und \texttt{-{}-max-cpu-count}}
Legen die minimale bzw. maximale anzahl der verwendbaren \glspl{CPU} fest.

\subsubsection{\texttt{-{}-min-ram} und \texttt{-{}-max-ram}}
Legen den minimal verfügbaren bzw. maximale verwendbaren Arbeitsspeicher fest.

\subsubsection{\texttt{-{}-server}}
Startet den Server, der die Aufgaben an die Arbeiter verteilt.

\section{Qualitätszielbestimmungen}
% Tabelle ... schafeln .. was ist ihm wichtig

\section{Globale Testfälle und Szenarien}
\subsection{Globale Testfälle}
\subsubsection{Grundlegende Testfälle}
\subsubsection{Erweiterte Testfälle}
\subsection{Szenarien}
\section{Entwicklungsumgebung}

\clearpage
\printnoidxglossaries

\end{document}
