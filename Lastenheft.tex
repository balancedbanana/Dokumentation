\documentclass[parskip=full]{scrartcl}
\usepackage[utf8]{inputenc}
\usepackage[T1]{fontenc}
\usepackage[german]{babel}
\usepackage{hyperref}
\hypersetup{pdftitle={Balanced Banana},bookmarks=true,}
\usepackage{graphicx}
\usepackage{csquotes}
\usepackage[nonumberlist]{glossaries}
\usepackage{enumitem}
\usepackage{verbatim}

\makeatletter
\newenvironment{mycode}
 {\def\@xobeysp{\ }\verbatim\rightskip=0pt plus 6em\relax}
 {\endverbatim}
\makeatother

\makenoidxglossaries

\newglossaryentry{CLI}
{
	name=CLI,
	description={Commandline Interface},
}

\title{Balanced Banana}
\subtitle{Distributed Task Scheduling}
\author{Niklas Lorenz \and Thomas Häuselmann \and Rakan Zeid Al Masri \and Christopher Lukas Homberger \and Jonas Seiler}

\begin{document}

\maketitle

\section{Einleitung}

\section{Zielbestimmung}

\section{Produkteinsatz}

\section{Produktumgebung}

\section{Funktionale Anforderungen}
\begin{itemize}[nosep]
\item[FA10] Automatische Aufgabenverteilung über ein \glspl{CLI} auf mehrere Rechner
\end{itemize}

\section{Produktdaten}
\begin{itemize}[nosep]
\item[PD10]
\end{itemize}

\section{Nichtfunktionale Anforderungen}
\begin{itemize}[nosep]
\item[NF10]
\end{itemize}

\section{Systemmodelle}

\subsection{Szenarien}

\subsection{Anwendungsfälle}

\section{Benutzer Oberfläche}
\subsection{\glspl{CLI}}
\begin{mycode}
bb [--block|-b] [<--priority|-p> <Integer>] [<--docker-file|-df> <path to dockerfile>] [<--max-cpu-count|-Mc> <Integer>] [<--min-cpu-count|-mc> <Integer>] [<--max-ram|-Mr> <Integer>] [<--min-ram|-mr> <Integer>] <program> [args]+
\end{mycode}

\subsubsection{\texttt{-{}-block}}
Kehrt erst nach der Ausführung der Aufgabe zum Aufrufer zurück. Nützlich um voneinander abhängige Aufgaben, in einem Konsolen Skript, nacheinander auszuführen.

\printnoidxglossaries

\end{document}
