\documentclass[parskip=full]{scrartcl}
\usepackage[utf8]{inputenc}
\usepackage[T1]{fontenc}
\usepackage[german]{babel}
\usepackage{hyperref}
\hypersetup{pdftitle={Balanced Banana},bookmarks=true,}
\usepackage{graphicx}
\usepackage{csquotes}
\usepackage[nonumberlist]{glossaries}
\usepackage{enumitem}
\usepackage{verbatim}

\makeatletter
\newenvironment{mycode}
 {\def\@xobeysp{\ }\verbatim\rightskip=0pt plus 6em\relax}
 {\endverbatim}
\makeatother

\makenoidxglossaries

\newglossaryentry{CLI}
{
	name=CLI,
	description={Commandline Interface},
}

\newglossaryentry{CPU}
{
	name=CPU,
	description={Central Processing Unit, Kern jedes Rechners um Anwendungen auszuführen},
}

\title{Balanced Banana}
\subtitle{Distributed Task Scheduling}
\author{Niklas Lorenz \and Thomas Häuselmann \and Rakan Zeid Al Masri \and Christopher Lukas Homberger \and Jonas Seiler}

\begin{document}

\maketitle

\section{Einleitung}

\section{Zielbestimmung}

\section{Produkteinsatz}

\section{Produktumgebung}

\section{Funktionale Anforderungen}
\begin{itemize}[nosep]
\item[FA10] Automatische Aufgabenverteilung über ein \glspl{CLI} auf mehere Rechner
\end{itemize}

\section{Produktdaten}
\begin{itemize}[nosep]
\item[PD10]
\end{itemize}

\section{Nichtfunktionale Anforderungen}
\begin{itemize}[nosep]
\item[NF10]
\end{itemize}

\section{Systemmodelle}

\subsection{Szenarien}

\subsection{Anwendungsfälle}

\section{Benutzer Oberfläche}
\subsection{\glspl{CLI}}
\begin{mycode}
bb [--block|-b] [<--priority|-p> <Integer>] [<--docker-file|-df> <path to dockerfile>] [<--max-cpu-count|-Mc> <Integer>] [<--min-cpu-count|-mc> <Integer>] [<--max-ram|-Mr> <Integer>] [<--min-ram|-mr> <Integer>] <program> [args]+
\end{mycode}

\subsubsection{\texttt{-{}-block}}
Kehrt erst nach der Ausführung der Aufgabe zum Aufrufer zurück. Nützlich um voneinander abhängige Aufgaben, in einem Konsolen Skript, nacheinander auszuführen.

\subsubsection{\texttt{-{}-priority}}
Legt die Priorität der auszuführenden Aufgabe fest, gefolgt von low (0), medium (1), high (2), banana (3).
Es kann der Name bzw. die Zahl in der Klammer als Priorität verwendet werden.

\subsubsection{\texttt{-{}-docker-file}}
Legt sämtliche Eigenschaften eines Docker Abbildes, gefolgt vom Pfad einer benutzerdefinierten \href{https://docs.docker.com/engine/reference/builder/}{DockerFile}, fest.
Beispielsweise Benutzer ID, installierte Programme, auszuführende Umgebung.

\subsubsection{\texttt{-{}-min-cpu-count} und \texttt{-{}-max-cpu-count}}
Legen die minimale bzw. maximale anzahl der verwendbaren \glspl{CPU} fest.

\subsubsection{\texttt{-{}-min-ram} und \texttt{-{}-max-ram}}
Legen den minimal verfügbaren bzw. maximale verwendbaren Arbeitsspeicher fest.

\printnoidxglossaries

\end{document}
