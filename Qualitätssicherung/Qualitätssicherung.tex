\documentclass[a4paper,12pt]{article}
\usepackage{amssymb} % needed for math
\usepackage{amsmath} % needed for math
\usepackage[utf8]{inputenc} % this is needed for german umlauts
\usepackage[ngerman]{babel} % this is needed for german umlauts
\usepackage[T1]{fontenc}    % this is needed for correct output of umlauts in pdf
\usepackage[margin=2.5cm]{geometry} %layout
\usepackage{booktabs}
\usepackage[hidelinks]{hyperref}
\hypersetup{pdftitle={Balanced Banana},bookmarks=true,}
\usepackage{graphicx}
\usepackage{csquotes}
\usepackage[nonumberlist]{glossaries}
\usepackage{enumitem}
\usepackage{verbatim}
\usepackage{indentfirst} % Adds indent for the first paragraph after a {/section}


\deftranslation[to=ngerman]{Glossary}{\section{Stichwortverzeichnis}}

\makeatletter
\newenvironment{mycode}
 {\def\@xobeysp{\ }\verbatim\rightskip=0pt plus 6em\relax}
 {\endverbatim}
\makeatother

\title{Balanced Banana}
\author{Niklas Lorenz \and Thomas Häuselmann \and Rakan Zeid Al Masri \and Christopher Lukas Homberger \and Jonas Seiler}


%%%%%%%%%%%%%%%%%%%%%%%%%%%%%%%%%%%%%%%%%%%%%%%%%%%%%%%%%%%%%%%%%%%%%%
% Create a shorter version for tables. DO NOT CHANGE               	 %
%%%%%%%%%%%%%%%%%%%%%%%%%%%%%%%%%%%%%%%%%%%%%%%%%%%%%%%%%%%%%%%%%%%%%%
\newcommand\addrow[2]{#1 &#2\\ }

\newcommand\addheading[2]{#1 &#2\\ \hline}
\newcommand\tabularhead{\begin{tabular}{lp{13cm}}
\hline
	}

\newcommand\addmulrow[2]{ \begin{minipage}[t][][t]{2.5cm}#1\end{minipage}%
   &\begin{minipage}[t][][t]{8cm}
    \begin{enumerate} #2   \end{enumerate}
    \end{minipage}\\ }

\newenvironment{usecase}{\tabularhead}
{\hline\end{tabular}}

\usepackage{microtype}

\begin{document}
\pagenumbering{roman}
\begin{titlepage}
    \begin{center}
    
     \vspace*{0.8cm}
 
        \includegraphics[width=0.5\textwidth]{balancedbanana}
        \vspace*{1cm}
 
        \Huge
        \textbf{Balanced Banana}
 
        \vspace{0.5cm}
        \LARGE
        A Distributed Task Scheduling System
        
        \vspace{0.5 cm}
        \LARGE
        Implementierung
 
        \vspace{1.5cm}

        \large
        \textbf{Niklas Lorenz, Thomas Häuselmann, Rakan Zeid Al Masri, Christopher Lukas Homberger und Jonas Seiler}
 
        \vspace*{0.5cm}

        \textbf{\today}
 
       
        
 
    \end{center}
\end{titlepage}         % Deckblatt.tex laden und einfügen
\setcounter{page}{2}
\tableofcontents          % Inhaltsverzeichnis ausgeben
\clearpage
\pagenumbering{arabic}

\section{Einleitung}
\vspace{0.2cm}
Dieses Dokument dient als Übersicht der Qualitätssicherung. Erläutert werden sowohl unsere Methoden zur Qualitätssicherung als auch das Ergebnis dieser. Ebenfalls beinhaltet dieses Dokument eine Anleitung zum Aufsetzen und Benutzen des endgültigen Programms.
\section{Testabdeckung}
\subsection{Klassenüberdeckung}
\vspace{0.2cm}
Überdeckung, evtl Änderung über 3 Wochen, der Coverage der einzelnen Klassen und Methoden

\includegraphics[width=\textwidth]{CoverageD}
\subsection{Integrationstests}
\vspace{0.2cm}
Tests die wir benutzt haben um die ganzen Testfälle abzudecken und wie diese gelaufen sind
\clearpage
\section{Testen des Systems}
\subsection{Aufsetzen des Systems}
\vspace{0.2cm}
Welche Umgebung hatten wir, was haben wir im Endeffekt getestet (zb das aufsetzte Netz von Florian, eine Woche Testbetrieb)
\begin{itemize}[label={\textbullet}]
    \item Herunterladen des Programms unter\newline \texttt{https://github.com/balancedbanana/balancedbanana/releases}
    \item Einrichten der MySql Datenbank mit \newline
    \begin{mycode}
    cat balancedbanana.sql | sudo mysql
    \end{mycode}
    Die SQL Datei ist im Archiv unter \newline \texttt{share/balancedbanana/balancedbanana.sql} \newline
    zu finden.\newline
    Nun ist die Datenbank unter dem Schema balancedbanana installiert.
    In der MySQL-Konsole einen Datenbank Nutzer für balancedbanana anlegen mit folgenden MySQL-Befehlen:
    \begin{mycode}
    CREATE USER 'balancedbanana'@'localhost' IDENTIFIED BY 
    'balancedbanana';
    	
    GRANT ALL PRIVILEGES ON balancedbanana.* TO 'balancedbanana'@'localhost';
        
    FLUSH PRIVILEGES;
        
    exit
    \end{mycode}
    Das Passwort \texttt{balancedbanana} nach \texttt{IDENTIFIED BY} bei Bedarf anpassen und in \newline
    \texttt{share/balancedbanana/.bbs/appconfig.ini} \newline
    in Zeile \texttt{databasepassword} anpassen. \newline

\end{itemize}
\subsection{Konklusion}
Wie lief es, Leistung, entdeckte Bugs, etc.
\clearpage
\end{document}
